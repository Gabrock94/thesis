\documentclass[ twoside,openright,titlepage,numbers=noenddot,headinclude,
                footinclude=true,cleardoublepage=empty,abstractoff,%
                BCOR=5mm,paper=letter,fontsize=11pt,letterpaper,%
                american,%
                ]{scrreprt}

\usepackage[square,numbers]{natbib}
\usepackage{amsmath}
\usepackage{xspace} % to get the spacing after macros right
\usepackage{mparhack} % get marginpar right
\usepackage{fixltx2e} % fixes some LaTeX stuff
\usepackage[printonlyused,smaller]{acronym}
\usepackage[pdfpagelabels]{hyperref}
\usepackage{graphicx}
\usepackage{tabularx}
    \setlength{\extrarowheight}{3pt} % increase table row height
\usepackage{etoolbox}

% Thesis metadata.
\input{metadata}

\hypersetup{
    colorlinks=false,
    hidelinks,
    linktocpage=true,
    breaklinks=true,
    hypertexnames=true,
    pdftitle={\metadata{title}},
    pdfauthor={\textcopyright\ \metadata{author}, \metadata{school}},
    pdfsubject={},
    pdfkeywords={},
    pdfcreator={pdfLaTeX},
    pdfproducer={LaTeX with hyperref and classicthesis}
}

\usepackage[eulerchapternumbers,listings,%
            pdfspacing,%floatperchapter,%linedheaders,%
            beramono,eulermath,parts]{classicthesis}

% XeTeX fonts.
\usepackage{fontspec}
\setmainfont[
    Ligatures={Common,TeX},
    BoldFont={Crimson-Bold},
    ItalicFont={Crimson-Italic},
    Path={crimson/},
    ]{Crimson-Roman}
% \setsansfont[
%     Ligatures={Common,TeX},
%     Scale=0.8,
%     ]{Avenir Next}

% Version marker.
% Set "final" option to turn off marker.
\usepackage[time=false]{prelim2e}
\input{revision}
\renewcommand{\PrelimWords}{%
    rev. \Revision%
    }

% I don't know why classicthesis wants this. Something about hyperlinks?
\newcounter{dummy}

\begin{document}
\raggedbottom
%\renewcommand*{\bibname}{new name}
\pagenumbering{roman}
\pagestyle{plain}


% TITLE PAGE
\begin{titlepage}
	% if you want the titlepage to be centered, uncomment and fine-tune the line below (KOMA classes environment)
	\begin{addmargin}[-1cm]{-3cm}
    \begin{center}
        \large
        \hfill
        \vfill
        \begingroup
            \color{Violet}\spacedallcaps{Hardware and Software} \\
            \color{Violet}\spacedallcaps{for Approximate Computing} \\ \bigskip
        \endgroup
        \spacedlowsmallcaps{Adrian Sampson}
        \vfill
        foo \\ \medskip
        foo\ -- bar
        \vfill
    \end{center}
  \end{addmargin}
\end{titlepage}


% TITLE BACK
\thispagestyle{empty}
\hfill
\vfill
\noindent\metadata{author}: \textit{\metadata{title},}
\textcopyright\ TK
%\bigskip
% more stuff here


% DEDICATION
\cleardoublepage
\thispagestyle{empty}
%\phantomsection
\refstepcounter{dummy}
\pdfbookmark[1]{Dedication}{Dedication}
\vspace*{3cm}
\begin{center}
    \emph{Ohana} means family. \\
    Family means nobody gets left behind, or forgotten. \\ \medskip
    --- Lilo \& Stitch
\end{center}

\medskip
\begin{center}
    Dedicated to the loving memory of Rudolf Miede. \\ \smallskip
    1939\,--\,2005
\end{center}


% ABSTRACT
%\renewcommand{\abstractname}{Abstract}
\cleardoublepage
\pdfbookmark[1]{Abstract}{Abstract}
\begingroup
\let\clearpage\relax
\let\cleardoublepage\relax
\let\cleardoublepage\relax
\chapter*{Abstract}
Short summary of the contents in English\dots
\endgroup
\vfill


% ACKS
\cleardoublepage
\pdfbookmark[1]{Acknowledgments}{acknowledgments}
\begin{flushright}{\slshape
    We have seen that computer programming is an art, \\
    because it applies accumulated knowledge to the world, \\
    because it requires skill and ingenuity, and especially \\
    because it produces objects of beauty.} \\ \medskip
    --- \defcitealias{knuth:1974}{Donald E. Knuth}\citetalias{knuth:1974} \citep{knuth:1974}
\end{flushright}
\bigskip
\begingroup
\let\clearpage\relax
\let\cleardoublepage\relax
\let\cleardoublepage\relax
\chapter*{Acknowledgments}
Regarding the typography and other help, many thanks go to Marco
Kuhlmann, Philipp Lehman, Lothar Schlesier, Jim Young, Lorenzo
Pantieri and Enrico Gregorio\footnote{Members of GuIT (Gruppo
Italiano Utilizzatori di \TeX\ e \LaTeX )}, J\"org Sommer,
Joachim K\"ostler, Daniel Gottschlag, Denis Aydin, Paride
Legovini, Steffen Prochnow, Nicolas Repp, Hinrich Harms,
 Roland Winkler, J\"org Weber,
 and the whole \LaTeX-community for support, ideas and
 some great software.
\endgroup


% CONTENTS
\pagestyle{scrheadings}
%\phantomsection
\refstepcounter{dummy}
\pdfbookmark[1]{\contentsname}{tableofcontents}
\setcounter{tocdepth}{2} % <-- 2 includes up to subsections in the ToC
\setcounter{secnumdepth}{3} % <-- 3 numbers up to subsubsections
\manualmark
\markboth{\spacedlowsmallcaps{\contentsname}}{\spacedlowsmallcaps{\contentsname}}
\tableofcontents
\automark[section]{chapter}
\renewcommand{\chaptermark}[1]{\markboth{\spacedlowsmallcaps{#1}}{\spacedlowsmallcaps{#1}}}
\renewcommand{\sectionmark}[1]{\markright{\thesection\enspace\spacedlowsmallcaps{#1}}}
\clearpage
\begingroup
    \let\clearpage\relax
    \let\cleardoublepage\relax
    \let\cleardoublepage\relax

    % Figures
    %\phantomsection
    \refstepcounter{dummy}
    %\addcontentsline{toc}{chapter}{\listfigurename}
    \pdfbookmark[1]{\listfigurename}{lof}
    \listoffigures

    \vspace*{8ex}

    % Tables
    \refstepcounter{dummy}
    %\addcontentsline{toc}{chapter}{\listtablename}
    \pdfbookmark[1]{\listtablename}{lot}
    \listoftables
    \vspace*{8ex}
\endgroup

\cleardoublepage


% PART 1
\pagenumbering{arabic}
%\setcounter{page}{90}
% use \cleardoublepage here to avoid problems with pdfbookmark
\cleardoublepage
\part{Some Kind of Manual}
\chapter{Introduction}
\label{ch:introduction}
A bunch of text will go here.


% APPENDIX
\appendix
\cleardoublepage
\part{Appendix}
\chapter{Appendix Test}
An appendix could go here.


% BIB
% work-around to have small caps also here in the headline
\manualmark
\markboth{\spacedlowsmallcaps{\bibname}}{\spacedlowsmallcaps{\bibname}} % work-around to have small caps also
%\phantomsection
\refstepcounter{dummy}
\addtocontents{toc}{\protect\vspace{\beforebibskip}} % to have the bib a bit from the rest in the toc
\addcontentsline{toc}{chapter}{\tocEntry{\bibname}}
\bibliographystyle{plainnat}
\label{app:bibliography}
\bibliography{thesis}


% COLOPHON
\pagestyle{empty}
\hfill
\vfill
\pdfbookmark[0]{Colophon}{colophon}
\section*{Colophon}
This document was typeset using the typographical look-and-feel \texttt{classicthesis} developed by Andr\'e Miede.
The style was inspired by Robert Bringhurst's seminal book on typography ``\emph{The Elements of Typographic Style}''.
\texttt{classicthesis} is available for both \LaTeX:
\begin{center}
\url{http://code.google.com/p/classicthesis/}
\end{center}
Happy users of \texttt{classicthesis} usually send a real postcard to the author, a collection of postcards received so far is featured here:
\begin{center}
\url{http://postcards.miede.de/}
\end{center}
\bigskip
\noindent\finalVersionString


% DECLARATION
\refstepcounter{dummy}
\pdfbookmark[0]{Declaration}{declaration}
\chapter*{Declaration}
\thispagestyle{empty}
Put your declaration here.
\bigskip
\noindent\textit{who knows what goes here}
\smallskip
\begin{flushright}
    \begin{tabular}{m{5cm}}
        \\ \hline
        \centering\metadata{author} \\
    \end{tabular}
\end{flushright}

\end{document}
