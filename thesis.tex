\documentclass[ twoside,openright,titlepage,numbers=noenddot,headinclude,
                footinclude=true,cleardoublepage=empty,abstractoff,%
                BCOR=5mm,paper=letter,fontsize=11pt,letterpaper,%
                american,%
                ]{scrreprt}

\usepackage[backend=biber,natbib=true,style=numeric-comp,%
            maxbibnames=99]{biblatex}
\usepackage{amsmath}
\usepackage{amssymb}
\usepackage{xspace} % to get the spacing after macros right
\usepackage{mparhack} % get marginpar right
\usepackage{fixltx2e} % fixes some LaTeX stuff
\usepackage[printonlyused,smaller]{acronym}
\usepackage[pdfpagelabels]{hyperref}
\usepackage{graphicx}
\usepackage{tabularx}
\usepackage{etoolbox}
\usepackage{listings}
\usepackage[eulerchapternumbers,parts,linedheaders]{classicthesis}
\usepackage{tikz}
\usepackage{import}
\usepackage{multirow}
\usepackage{amsthm}
\usepackage[shorthand,inference]{semantic}  % EnerJ proofs.
\usepackage{supertabular}
\usepackage{subcaption}
\usepackage{mathpartir}
\usepackage{csquotes}
\usepackage{amsfonts}
\usepackage{color}
\usepackage{chngcntr}
\usepackage[update,prepend]{epstopdf}
\usepackage[algoruled,linesnumbered,noline]{algorithm2e}

% Thesis metadata.
\input{metadata}

\hypersetup{
    colorlinks=false,
    hidelinks,
    linktocpage=true,
    breaklinks=true,
    hypertexnames=true,
    pdftitle={\metadata{title}},
    pdfauthor={\textcopyright\ \metadata{author}, \metadata{school}},
    pdfsubject={},
    pdfkeywords={},
    pdfcreator={pdfLaTeX},
    pdfproducer={LaTeX with hyperref and classicthesis}
}

% XeTeX fonts.
\usepackage{fontspec}
\setmainfont[
    Ligatures={Common,TeX},
    UprightFont={*-Roman},
    BoldFont={*-Bold},
    ItalicFont={*-Italic},
    BoldItalicFont={*-BoldItalic},
    Extension={.otf},
    Path={crimson/},
]{Crimson}
\setmonofont[
    UprightFont={*-Regular},
    BoldFont={*-Bold},
    Extension={.ttf},
    Path={inconsolata/},
    Scale=MatchLowercase,
]{Inconsolata}
\newfontfamily\noligz[
    Ligatures={NoCommon},
    UprightFont={*-Roman},
    BoldFont={*-Bold},
    ItalicFont={*-Italic},
    BoldItalicFont={*-BoldItalic},
    Extension={.otf},
    Path={crimson/},
]{Crimson}

% Version marker.
% Set "final" option to turn off marker.
\usepackage[time=false]{prelim2e}
\input{revision}
\renewcommand{\PrelimWords}{%
    rev. \Revision%
    }

% Listings.
\lstset{
    language=Java,
    emph={@Approx, @Precise, @Top, @Context, endorse, @Approximable, assert},
    emphstyle=\bfseries,
    columns=fullflexible,
    basicstyle=\ttfamily,
    aboveskip=1mm plus 1mm minus 1mm,
    belowskip=1mm plus 1mm minus 1mm,
    mathescape=true,
}
\lstloadlanguages{Java,C,C++,Python}
\newcommand{\code}{\lstinline[emphstyle={},keywordstyle={}]}
\newcommand{\ilcode}[1]{\code!#1!}
\newcommand{\lil}{\code}

% I don't know why classicthesis wants this. Something about hyperlinks?
\newcounter{dummy}

% For EnerJ (proofs).
\newtheorem{definition}{Definition}[section]
\newtheorem{lemma}[definition]{Lem\-ma}
\newtheorem{assumption}[definition]{Assumption}
\newtheorem{theorem}[definition]{Theorem}
\newtheorem{corollary}[definition]{Corollary}
\newcommand\proofcase[1]{\vspace{4mm plus 1mm minus 1mm}\noindent\textbf{#1}}
\def\qed{\unskip\kern 10pt{\unitlength1pt\linethickness{.4pt}\framebox(6,6){}}}

\newcommand{\bench}[1]{\textsf{#1}}

% For the semantics.
\newcommand{\defeq}{\equiv}
\newcommand{\judge}{\Downarrow}
\newcommand{\alt}{\:|\:}

% Math macros for probability stuff.
\newcommand{\prob}[1]{\ensuremath{\Pr\!\left[#1\right]}}
\newcommand{\expc}[1]{\ensuremath{\mathrm{E}\!\left[#1\right]}}
\newcommand{\varc}[1]{\ensuremath{\mathrm{Var}\!\left[#1\right]}}

\hyphenation{analy-ses analy-sis proto-type}

% ToC fonts!
\renewcommand\cftpartfont{\normalfont\color{Violet}}
\renewcommand\cftpartpagefont{\normalfont\color{Violet}}

\newcommand{\etal}{et~al.\@\xspace}
\newcommand{\TODO}[1]{\textcolor{red}{{\bf TODO:} #1}}

\newcommand{\chref}[1]{Chapter~\ref{ch:#1}\xspace}
\newcommand{\plurchref}[1]{Chapters~\ref{ch:#1}\xspace}
\newcommand{\secref}[1]{Section~\ref{sec:#1}}
\newcommand{\figref}[1]{Figure~\ref{sec:#1}}
\newcommand{\tabref}[1]{Table~\ref{sec:#1}}

% Using this instead of \addbibresource so that the Makefile picks up the
% bib filename.
\bibliography{thesis}

% Disable ugly quotation marks around titles in bibliography.
\DeclareFieldFormat
  [article,inbook,incollection,inproceedings,patent,thesis,unpublished]
  {title}{#1\isdot}

% Don't use a colon after "In" before booktitle.
\renewcommand*{\intitlepunct}{\space}

% Use a comma between booktitle and year, not a dot.
\DeclareFieldFormat
  [inproceedings]
  {booktitle}{\mkbibemph{#1}\addcomma}

% Misc entries: minimal formatting.
\DeclareFieldFormat
  [misc]
  {title}{#1\isdot}

\begin{document}
\raggedbottom
\renewcommand*{\bibname}{References}
\pagenumbering{roman}
\pagestyle{plain}


% TITLE PAGE
\begin{titlepage}
	% if you want the titlepage to be centered, uncomment and fine-tune the line below (KOMA classes environment)
	\begin{addmargin}[-1cm]{-3cm}
    \begin{center}
        \large
        \hfill
        \vfill
        \begingroup
            \color{Violet}\spacedallcaps{Hardware and Software} \\
            \color{Violet}\spacedallcaps{for Approximate Computing} \\ \bigskip
        \endgroup
        \spacedlowsmallcaps{Adrian Sampson}
        \vfill
        A dissertation \\
        submitted in partial fulfillment of the \\
        requirements for the degree of \\
        \medskip
        \metadata{degree} \\
        \bigskip
        \metadata{school} \\
        \smallskip
        \metadata{year}
        \vfill
        Reading Committee: \\
        \smallskip
        Luis Ceze, Chair \\
        \smallskip
        Dan Grossman, Chair \\
        \smallskip
        Mark Oskin
        \vfill
        Program Authorized to Offer Degree: \\
        \smallskip
        \metadata{department}
    \end{center}
  \end{addmargin}
\end{titlepage}


% COPYRIGHT PAGE
\cleardoublepage
\thispagestyle{empty}
\begin{center}
    \vspace*{20ex}
    \textcopyright\ \metadata{year} \\
    \bigskip
    \metadata{author}
\end{center}


% ABSTRACT
\cleardoublepage
\thispagestyle{empty}
\pdfbookmark[1]{Abstract}{Abstract}

\begin{center}
    \spacedlowsmallcaps{Abstract} \\
    \bigskip
    {\color{Violet}
    \spacedallcaps{Hardware and Software} \\
    \spacedallcaps{for Approximate Computing} \\
    }
    \bigskip
    \metadata{author} \\
    \bigskip
    Chairs of the Supervisory Committee: \\
    Associate Professor Luis Ceze \\
    Associate Professor Dan Grossman \\
    \metadata{department} \\
    \bigskip
\end{center}

\subimport*{text/}{abstract}


% DEDICATION
\cleardoublepage
\thispagestyle{empty}
%\phantomsection
\refstepcounter{dummy}
\pdfbookmark[1]{Dedication}{Dedication}
\vspace*{3cm}
\begin{center}
    The ordered swirl of houses and streets, from this high angle, sprang at
    her now with the same unexpected, astonishing clarity as the circuit card
    had\ldots\ %
    % Though she knew even less about radios than about Southern Californians,
    % there were to both outward patterns a hieroglyphic sense of concealed
    % meaning, of an intent to communicate.
    There'd seemed no limit to what the printed circuit could have told her
    (if she had tried to find out); so in her first minute of San Narciso, a
    revelation also trembled just past the threshold of her understanding.
    \\ \medskip
    --- Thomas Pynchon, \textit{The Crying of Lot 49}
\end{center}


% CONTENTS
\renewcommand{\contentsname}{Table of Contents}
\pagestyle{scrheadings}
%\phantomsection
\refstepcounter{dummy}
\pdfbookmark[1]{\contentsname}{tableofcontents}
\setcounter{tocdepth}{1} % <-- 2 includes up to subsections in the ToC
\setcounter{secnumdepth}{3} % <-- 3 numbers up to subsubsections
\manualmark
\markboth{\spacedlowsmallcaps{\contentsname}}{\spacedlowsmallcaps{\contentsname}}
\tableofcontents
\automark[section]{chapter}
\renewcommand{\chaptermark}[1]{\markboth{\spacedlowsmallcaps{#1}}{\spacedlowsmallcaps{#1}}}
\renewcommand{\sectionmark}[1]{\markright{\thesection\enspace\spacedlowsmallcaps{#1}}}
\clearpage

\cleardoublepage


% ACKS
\cleardoublepage
\pdfbookmark[1]{Acknowledgments}{acknowledgments}
\chapter*{Acknowledgments}
\subimport*{text/}{acks}


\pagenumbering{arabic}
\cleardoublepage


\part{Approximate Computing}
\label{part:overview}

\chapter{Overview}
\label{ch:overview}
\subimport*{text/}{overview}

\chapter{Survey}
\label{ch:related}
\subimport*{text/}{related}


\part{Programmable Approximation}
\label{part:programming}

\chapter{A Safe and General Language Abstraction}
\label{ch:enerj}
\subimport*{papers/enerj/}{epaj}


\chapter{Probability Types}
\label{ch:decaf}
\subimport*{papers/decaf/}{approx-within}


\chapter{Probabilistic Assertions}
\label{ch:passert}
\subimport*{papers/passert/}{paper}



\part{Approximate Systems}
\label{part:systems}

\chapter{Approximate Storage}
\label{ch:approxstorage}
\subimport*{papers/approxstorage/}{approxstorage}


\chapter{An Open-Source Approximation Infrastructure}
\label{ch:accept}
\subimport*{papers/accept/}{accept}


\part{Closing}
\label{part:conclusion}

\chapter{Retrospective}
\label{ch:conclusion}
\subimport*{text/}{conclusion}

\chapter{Prospective}
\label{ch:future}
\subimport*{text/}{future}


% BIB
% work-around to have small caps also here in the headline
\manualmark
\markboth{\spacedlowsmallcaps{\bibname}}{\spacedlowsmallcaps{\bibname}} % work-around to have small caps also
%\phantomsection
\refstepcounter{dummy}
\addtocontents{toc}{\protect\vspace{\beforebibskip}} % to have the bib a bit from the rest in the toc
\addcontentsline{toc}{chapter}{\tocEntry{\bibname}}
\label{app:bibliography}
\printbibliography


\appendix
\part{Appendix: Semantics and Theorems}

\chapter{EnerJ: Noninterference Proof}
\label{app:enerj}
\subimport*{papers/enerj/}{ott_epaj_proofs}

\chapter{Probability Types: Soundness Proof}
\label{app:decaf}
\subimport*{papers/decaf/semantics/}{semantics}

\chapter{Probabilistic Assertions: Equivalence Proof}
\label{app:passert}
\subimport*{papers/passert/}{appendix}


% COLOPHON
\cleardoublepage
\pagestyle{empty}
\vspace*{20ex}
\pdfbookmark[0]{Colophon}{colophon}
\section*{Colophon}
This thesis was designed using Andr\'e Miede's
\texttt{classicthesis} style for \LaTeX.
The body, titles, and math are set in Crimson Text, a typeface by Sebastian Kosch.
Code appears in \texttt{Inconsolata} by Raph Levien, which is based on Luc(as)
de Groot's Consolas.
Both typefaces are open-source projects.



\end{document}
