\section{Overview}

% The primary goal of \sysname is to enable safe, practical, and intuitive
% approximation for general-purpose programs on commodity hardware.  Like previous
% approximation systems~\cite{enerj, relax, flikker}, \sysname involves the programmer in
% application-dependent decisions about where approximation can apply: the
% programmer provides annotations, \sysname highlights opportunities where it can apply program relaxations to improve performance, and possibly offers suggestions to the programmer on code changes that would enable more approximation opportunities. 

To safely and efficiently harness the potential of approximate programs,
\sysname combines three main techniques: (1) a programmer--compiler
feedback loop consisting of source code annotations and an analysis log; (2) a
compiler analysis library
that enables a range of automatic program relaxations; and (3) an autotuning
system that uses dynamic measurements of candidate program relaxations to find
the best balances between efficiency and quality. The final output is a set of
Pareto-optimal versions of the input program that reflect its
efficiency--quality trade-off space.

Figure~\ref{fig:overview} illustrates
how these components make up \sysname's workflow.
Two feedback loops control the impact of
potentially destructive program relaxations: a \emph{static} feedback loop
providing
conservative guarantees
and a complementary \emph{dynamic} feedback loop that measures real
program behavior to choose the best optimizations.
%
A key hypothesis of this paper is that neither static nor dynamic constraints
are sufficient, since dynamic measurements cannot offer guarantees and static
constraints do not capture the full complexity of relationships among
relaxations, performance, and output quality. Together, however, the two
feedback loops make \sysname's optimizations both controlled and practical.
%
%\sysname's workflow comprises three main components:

\paragraph{Safety constraints and feedback.}
Because program relaxations can have outsize effects on program behavior,
programmers need \emph{visibility} into---and \emph{control} over---the
transformations the compiler applies.
%
To give the programmer fine-grained control over relaxations, \sysname extends
an existing lightweight annotation system for approximate computing based on
type qualifiers~\cite{enerj}.
%
\sysname gives programmers visibility into the relaxation process via feedback
that identifies which transformations can be applied and which annotations are
constraining it.  Through annotation and feedback, the programmer iterates
toward an annotation set that unlocks new performance benefits while relying on
an assurance that critical computations are unaffected.

\paragraph{Automatic program transformations.}
Based on programmer annotations, \sysname's compiler passes apply
transformations that involve only approximate data.
To this end,
\sysname provides a common analysis library that identifies code regions
that can be safely transformed.
We bring \sysname's safety analysis, programmer feedback, and automatic site
identification to existing work
on approximate program transformations~\cite{perforation, races-ibm,
rinard-hotpar, quickstep, dubstep, npu, anpu}.

\paragraph{Autotuning.}
While a set of annotations may permit many different safe program relaxations,
not all of them are beneficial.
A practical system must help programmers choose from
among many candidate relaxations for a given program to strike an
optimal balance between performance and quality.
\sysname's autotuner heuristically explores the space of possible relaxed
programs to identify Pareto-optimal variants.

