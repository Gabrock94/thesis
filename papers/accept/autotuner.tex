\section{Autotuning Search}
\label{sec:autotuner}

The autotuner is a test harness in which \sysname explores the space of possible
program relaxations through empirical feedback.  We call
a particular selection of relaxations and associated parameters (e.g., loop
perforation with factor $p$) a \emph{relaxation configuration}.  The
autotuner heuristically generates relaxation configurations and identifies the
ones that best balance performance and output quality.
%
The programmer also provides multiple inputs to the program.  \sysname validates
relaxation configurations by running them on fresh inputs to avoid overfitting.

Because the definition of quality is application dependent, \sysname relies on
programmer-provided \emph{quality metrics} that measure output
accuracy, as in previous work~\cite{enerj, truffle, qosprof, carbin-pldi, green,
npu}.
The quality metric is another program that (1) reads the outputs
from two different executions of the program being transformed and (2) produces
an error score between 0.0 (outputs are identical) and 1.0 (outputs are
completely different), where the definitions of ``identical'' and ``different''
are application dependent.

A na\"ive method of exploring the space of relaxation configurations is to
enumerate all possible configurations.
But the space of possible relaxation configurations is exponential in the number
of relaxation opportunities and therefore infeasible to even enumerate, let
alone evaluate empirically.
We instead use a heuristic that prioritizes a limited number of
executions that are likely to meet a minimum output quality.

\sysname's heuristic configuration search consists of two steps: it vets each
relaxation opportunity individually and then composes relaxations to create
composites.

\paragraph{Vetting individual relaxations.}
In the first step, the autotuner separately evaluates each
relaxation opportunity \sysname's analysis identified. Even with \sysname's
static constraints, it is
possible for some relaxations to lead to unacceptably degraded output or
zero performance benefit.
When the programmer uses escape hatches such as \lil{ENDORSE}
incorrectly, approximation can affect control flow or even pointers and
hence lead to crashes.
\sysname vets each
relaxation opportunity to disqualify unviable or unprofitable ones.

For each relaxation opportunity, the autotuner executes the program
with only that relaxation enabled.
If the output error is above a threshold, the running time averaged over
several executions is slower than the baseline,
or the program crashes, the relaxation is discarded.
Then, among the surviving relaxations, the autotuner increases the
aggressiveness of any optimizations that have parameters.
(In our prototype, only loop perforation has a variable
parameter: the perforation factor $p$.)
The autotuner records the range of parameters for which each opportunity site
is ``good''---when its error is below a threshold and it offers
speedup over the original program---along with the running time and
quality score.
These parameters are used in the next step to create composite configurations.

\paragraph{Composite configurations.}

After evaluating each relaxation opportunity site individually, \sysname's
autotuner composes multiple relaxations to produce the best overall
program configurations. For a program of even moderate size, it is infeasible to
try every possible combination of component relaxations.
\sysname heuristically predicts which combinations will yield the
best performance for a given quality constraint and validates only the best
predictions experimentally.

To formulate a heuristic, \sysname
hypothesizes that
relaxations compose linearly. That is, we assume that two program relaxations
that yield output error rates $e_1$ and $e_2$, when applied simultaneously,
result in an error of $e_1 + e_2$ (and that performance
will compose similarly).
Different relaxations can in practice compose unpredictably,
but this simplifying assumption is a tractable approximation that \sysname
later validates with real executions.

The configuration-search problem
is equivalent to
the 0/1 Knapsack Problem.
In the
Knapsack formulation, each configuration's output error is its \emph{weight}
and its performance benefit $1 - \frac{1}{\text{speedup}}$ is its
\emph{value}. The goal is to find the configuration that provides
the most total value subject to a maximum weight capacity.

The Knapsack Problem is NP-complete and intractable
even for programs with only a few dozen potential relaxations.
Instead, \sysname uses a well-known approximation algorithm~\cite{knapsack}
to sort the configurations by
their value-to-weight ratio and greedily selects configurations in rank order
up to an error budget.
To account for our simplifying assumptions, we use a range of error budgets to
produce multiple candidate composites.
The
algorithm is dominated by the sorting step, so its running time is
$O(n \log n)$ in the number of vetted relaxation-opportunity sites (and
negligible in practice).
%
Like other candidate configurations, the composites are executed
repeatedly to measure their true output quality and speedup.
