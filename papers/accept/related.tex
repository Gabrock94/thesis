\section{Related Work}
\label{accept:sec:related}

\sysname builds on a body of prior work on approximate
computing. Three main research directions are most
relevant to \sysname: static safety analyses for approximate programs,
program relaxations, and quality-aware autotuners.

One group of proposals seeks to statically analyze approximate programs to
prove properties even in the face of unreliable hardware or program
transformations.
Carbin \etal propose a general proof system for relating baseline executions
to relaxed executions~\cite{carbin-pldi}
and an integrity property checker for loop perforation~\cite{carbin-pepm}.
Rely~\cite{rely} and Chisel~\cite{chisel} analyze and tune the chance that a
nondeterministic computation goes wrong.
EnerJ~\cite{enerj} provides a simple noninterference guarantee that we adapt
in this work.
Other recent
work~\cite{sasa-sas11, zhu-popl12} uses probabilistic reasoning to prove
conservative accuracy guarantees for relaxations of restricted programming
patterns.
We instead focus on bringing approximation to general programs and rely on
programmer involvement and dynamic testing to build confidence in an
approximation's suitability.
\sysname's safety guarantees need to be both general and lightweight,
requiring minimal programmer overhead, to be practical. The information-flow
types adapted from EnerJ permit straightforward automated compiler
reasoning and are not constrained to specific code patterns.
%
EnerJ itself, however, is limited to a specific style of approximation, where
individual variables and instructions introduce error on hypothetical future
hardware.
\sysname expands the scope to today's computers using compiler analysis and
programmer feedback to enable coarser-grained optimizations.

\sysname complements specific software approximation strategies,
such as loop perforation~\cite{perforation},
alternate-implementation selection~\cite{petabricks, green, taco-soc}, parameter
selection~\cite{dynamicknobs}, synchronization relaxation~\cite{quickstep,
dubstep, races-ibm, rinard-hotpar}, and pattern
substitution~\cite{paraprox}. This work contributes a
unifying framework to make relaxations controlled and
automatic.

\sysname's autotuning component resembles other prior work on dynamic
measurement of approximation strategies, including off-line tools such as
PetaBricks~\cite{ansel-autotuning, petabricks} and on-line methods
such as Green~\cite{green}, ApproxIt~\cite{approxit}, and SAGE~\cite{sage}.
Misailovic
\etal's quality-of-service profiler~\cite{qosprof} evaluates the results of
candidate loop perforations by assessing their performance and quality impacts
for reporting to the programmer.
Precimonious~\cite{precimonious} tunes variables' floating-point widths to
adjust overall precision.
%
\sysname generalizes these autotuning
techniques: it applies to a variety of safety-constrained
relaxations and guides the process using a practical search
heuristic.

The type-qualifier overlay system we develop for Clang and LLVM is modeled
after work on practical pluggable types for Java~\cite{jsr308, papi}.
%
Another notable predecessor is Cqual~\cite{cqual}, which provided a similar mechanism
in GCC for finding bugs (e.g., format-string
vulnerabilities~\cite{shankar:typequal}).
\sysname's implementation of type qualifiers adds integration with the rest of
the compiler toolchain via IR metadata, which enables program analysis beyond
type checking and type-based optimization.

Our neural accelerator design builds on recent work on using hardware neural
networks as accelerators~\cite{npu, anpu, temam-isca13}.
To our knowledge, this is the first work
to demonstrate safe, automated, compiler-based offloading to the neural network.
