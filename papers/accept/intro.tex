\section{Introduction}\label{sec:intro}

% Energy is an increasingly important consideration for system designers;
% conserving energy enables longer battery life and lower chip temperatures.
%
Recent work on \emph{approximate computing} has exploited the fact that 
many applications, particularly those whose outputs are meant for human
interpretation, can enable more efficient execution at the cost of slightly
inaccurate outputs. 3-D rendering, search, and machine learning
are examples of the many tasks that tolerate inaccuracy.

Research over the past few years has proposed a variety of software and hardware 
approaches to approximate computing, including:
%
producing multiple versions of a program with varying accuracy and choosing
between them at run
time~\cite{green, dynamicknobs};
%
periodically skipping loop iterations~\cite{perforation, qosprof};
%
removing synchronization to reduce contention~\cite{dubstep, quickstep, races-ibm};
%
adjusting floating-point precision~\cite{precimonious};
%
using special low-power hardware
structures that produce wrong results probabilistically~\cite{truffle, flikker};
%
and training hardware neural networks to mimic the behavior of costly
functions~\cite{npu, anpu}.

These proposals share an important distinction
from traditional program optimizations:
they have subtle and broad-ranging effects on safety, reliability, and output
quality.
Some work relies on programmers for manual reasoning to control
these
effects~\cite{npu, flikker, races-ibm, perforation},
while other work proposes automated transformation based on code
patterns or exhaustive search~\cite{green, paraprox, sage}.
Manual code editing can be tedious and error-prone, especially since
important safety invariants are at stake.
Conversely, full automation eliminates a crucial element of
visibility and control.
Programmers must trust the automated system; they have no recourse when
opportunities are missed or invariants are broken.

We propose \sysname
(an Approximate C Compiler for Energy and Performance Trade-offs),
a framework for approximation that balances automation with programmer
guidance.
\sysname is \emph{controlled} because it preserves programmer intention
expressed via code annotations.
A static analysis rules out
unintended side effects. The programmer participates in a feedback loop with
the analysis to enable more approximation
opportunities.
\sysname is \emph{practical} because it facilitates a range
of approximation techniques that work on currently available hardware.
Just as a traditional compiler framework provides common tools to support
optimizations,
\sysname's building blocks help implement automatic
approximate transformations based on programmer guidance and dynamic feedback.

\sysname's architecture combines static and dynamic components.
The frontend, built atop LLVM~\cite{llvm}, extends the syntax of C and C++ to
incorporate an \lil{APPROX} keyword that programmers use to annotate types as
in other work~\cite{enerj}.
\sysname's central analysis, \emph{\precisepurity}, identifies coarse-grained
regions of code that can
affect only approximate values.
Coarse region selection is crucial for safe approximation strategies:
client optimizations use the results to
transform code and offload to accelerators while preserving static safety
properties.
After compilation, an autotuning component measures program executions
and uses heuristics to identify program variants that maximize performance and output quality.
To incorporate application insight, \sysname furnishes programmers with feedback
to guide them toward better annotations.


\paragraph{Contributions.}
\sysname is an end-to-end framework that makes existing proposals for approximate
program transformations practical and disciplined.
Its contributions are:
\begin{itemize}
\item A programming model for program relaxation that combines lightweight
annotations with compiler analysis feedback
to guide programmers toward effective relaxations;

\item An autotuning system that efficiently searches for a program's
best approximation parameters;

\item A core analysis library that identifies code that can be safely relaxed
or offloaded to an approximate accelerator;

\item A prototype implementation demonstrating both pure-software optimizations
and hardware acceleration using an off-the-shelf FPGA part.
\end{itemize}

\noindent
We evaluate \sysname across three platforms:
a standard Intel-based server;
a mobile SoC with an on-chip FPGA, which we use as an approximate accelerator;
and an ultra-low-power, energy-harvesting embedded microcontroller where
performance is critical to applications' viability.
The experiments demonstrate average speedups of \result{speedup-x86-mean},
\result{speedup-zynq-mean}, and \result{speedup-msp430-mean} on the
three platforms, respectively,
with quality loss under 10\%.

We also report qualitatively on the programming experience.
Novice C++ programmers were able to apply \sysname to
legacy software to obtain new speedups.
\sysname's combination of static analysis and dynamic measurement alleviates
much of the manual labor from the process of applying
approximation without sacrificing transparency or control.

The \sysname framework is open source and ready for use as research
infrastructure.
It provides the necessary language and compiler support to prototype and
evaluate new strategies for approximation,
reducing the need to reinvent these components for each new research
evaluation.
