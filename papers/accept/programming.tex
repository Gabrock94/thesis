\section{Annotation and Programmer Feedback}
\label{accept:sec:annotation-feedback}

This section describes \sysname's annotations and feedback,
which helps programmers balance safety with approximation.
Rather than proving theoretical accuracy guarantees for restricted programming
models as in other work~\cite{sasa-sas11, zhu-popl12, passert},
\sysname's workflow extends mainstream development practices: it combines
lightweight safety guarantees, programmer insight, and testing to apply
approximation to general code.

\subsection{Annotation Language}
\label{accept:sec:language}

The programmer uses annotations to communicate to the compiler which parts of
a program are safe targets for program relaxation.
\sysname adapts the type system of EnerJ, a language for approximate computing
that was originally developed for bounding the effects of unreliable hardware
components~\cite{enerj}.
As in EnerJ, all types in \sysname are qualified as either \emph{approximate} or
\emph{precise}, with precise being the default.

\paragraph{Information flow and endorsement}
\sysname's type system uses a standard information-flow
approach to prevent approximate data from affecting precise data.
%
Precise types are strict subtypes of their approximate counterparts, so
variables annotated as approximate cannot be assigned to precise ones without
explicit programmer intervention.  (The opposite direction, wherein precise data
affects approximate data, is permitted.)
%
\sysname's type system is directly derived from EnerJ's~\cite{enerj}, for which
its authors prove a noninterference property ensuring that precise data stays
precise.  The same noninterference guarantee applies to \sysname's
type-qualifier extension for type-safe subsets of C and C++.
Undefined behavior in C and C++ remains undefined in \sysname:
programs that violate type safety can also violate \sysname's guarantees.

Strict information flow can be too constraining.
For example, a program may need to
compute an approximate value, where relaxations can apply, but then check the
resulting output for integrity while treating it as precise:
%
\begin{lstlisting}
APPROX int a = expensiveCall();
cheapChecksumPrecise(a); // illegal
\end{lstlisting}
%
To permit this pattern, \sysname provides an \emph{endorsement} expression
that acts as a cast from an approximate type to its precise equivalent.
The above program fails to typecheck, but using \lil{ENDORSE(a)} as the argument
is legal:
\begin{lstlisting}
APPROX int a = expensiveCall();
cheapChecksumPrecise(ENDORSE(a));
\end{lstlisting}
Endorsements give programmers explicit control over information flow
when dealing with approximate values.

\paragraph{Pointer types}
For basic, non-reference types, \sysname's dialect of C allows unidirectional
information flow: precise values can be assigned into approximate variables
but not vice-versa. For pointers and references, however,
even precise-to-approximate flow is unsound since
it creates aliases for the same data that disagree on its type.
Pointer types are therefore invariant in the referent type.
The language does not permit approximate pointers---i.e., addresses must be
precise.

\paragraph{Implicit flow}
Control flow provides an avenue for approximate data to affect precise data
without a direct assignment. For example, \lil{if (a) p = 5;} allows the
variable \lil{a} to affect the value of \lil{p}.
Like EnerJ, \sysname prohibits approximate values from being used in
conditions---specifically, in \lil{if}, \lil{for}, \lil{do}, \lil{while}, and
\lil{switch} statements and in the ternary conditional-expression operator.
Programmers can use endorsements to explicitly circumvent this restriction.

\paragraph{Escape hatches}
\sysname decides whether program relaxations are safe based on the
\emph{effects} of the statements involved. Section~\ref{accept:sec:relaxations} goes
into more detail, but at a high level, code can be relaxed if its externally
visible effects are approximate.  For example, if \lil{a} is a pointer to
an \lil{APPROX int}, then the statement \lil{*a = 5;} has an approximate effect
on the heap.
Escape hatches from this sound reasoning are critical in a practical
system that must handle legacy code.
To enable or disable specific optimizations, the programmer can
override the compiler's decision about a statement's effects using two
annotations. The \annpermit annotation forces a statement to
be considered approximate
and \annforbid forces it to be precise, forbidding
any relaxations involving it.

These two annotations represent escape hatches from \sysname's normal
reasoning and thus violate the safety guarantees it normally provides.
%
Qualitatively, when annotating programs, we use these
annotations much less frequently than the primary annotations
\lil{APPROX} and \lil{ENDORSE}. We find \annpermit to be
useful when experimentally exploring program behavior before annotating and in
system programming involving memory-mapped registers.  Conversely,
\annforbid is useful for marking parts of the program involved in
introspection. Section~\ref{accept:sec:casestudy} gives more detail on these
experiences.

\subsection{Programmer Feedback}
\label{accept:sec:feedback}

\sysname takes inspiration from parallelizing compilers that use a development
feedback loop to help guide the programmer toward parallelization
opportunities~\cite{canal, deditor}.
It provides
feedback through an \term{analysis log} that describes the relaxations that it
attempted to apply. For example, for \sysname's synchronization-elision
relaxation, the log lists every lexically scoped lock acquire/release pair in
the program. For each relaxation opportunity, it reports whether the relaxation
is safe---whether it involves only approximate data---and, if it is
not, identifies the statements that prevent the relaxation from applying.
We call these statements with
externally visible precise effects \emph{blockers}.

\sysname reports blockers for each failed
relaxation-opportunity site. For example, during the annotation of one program
in our evaluation, \sysname examined this loop:
%
\begin{lstlisting}[numbers=left,firstnumber=650,numbersep=-1pt,numberstyle=\sffamily]
  double myhiz = 0;
  for (long kk=k1; kk<k2; kk++) {
    myhiz += dist(points->p[kk], points->p[0],
      ptDimension) * points->p[kk].weight;
  }
\end{lstlisting}
%
The store to the precise (by default) variable \lil{myhiz}
prevents the loop from being approximable. The analysis log reports:
%
\begin{lstlisting}
loop at streamcluster.cpp:651
blockers: 1
 * streamcluster.cpp:652: store to myhiz
\end{lstlisting}
%
Examining that loop in context, we found that \lil{myhiz} was a weight
accumulator that had little impact on the algorithm, so we changed its type from
\lil{double} to \lil{APPROX double}. On its next execution, \sysname logged the
following message about the same loop, highlighting a new relaxation
opportunity:
%
\begin{lstlisting}
loop at streamcluster.cpp:651
can perforate loop
\end{lstlisting}
%
The feedback loop between the programmer's annotations and the compiler's
analysis log strikes a balance with respect to programmer involvement: it helps
identify new relaxation opportunities while leaving the programmer in control.
%
Consider the alternatives on either end of the programmer-effort spectrum:
On one extreme, suppose that a programmer wishes to speed up a loop by manually
skipping iterations.  The programmer can easily misunderstand the loop's side
effects if it indirectly makes system calls or touches shared data.
%
On the other extreme, unconstrained automatic transformations are even more
error prone: a tool that removes locks can easily create subtle concurrency
bugs.
%
Combining programmer feedback with compiler assistance balances the
advantages of these approaches.
