\begin{theorem} \label{passert:thm:main}
Let $(0, H_0, p) \judge_s \{ x \}$, where $x$ is a finite program.
Then $(\Sigma, H_0, p) \judge_c b$ if and only if $(\Sigma, x) \judge_o b$.
\end{theorem}

Intuitively, this theorem is true because the distribution extraction
$\judge_s$ is just a call-by-need lazy evaluation, and $\judge_o$ is
the projection of $\judge_c$ over this lazy evaluation.  We prove the
theorem formally here.

The proof of this theorem proceeds by structural induction on $p$.
First, a few lemmas establish corresponding properties for
conditionals, expressions, then statements, and finally programs.

\begin{lemma} \label{passert:lem:expr}
  For $e \in E$, let $(H_s, e) \judge_s \{ x \}$, and suppose that for
  every variable $a$, $(\Sigma, H_s(a)) \judge_e H_c(a)$.  Then $(H_c,
  e) \judge_c v$ if and only if $(\Sigma, x) \judge_o v$.
\end{lemma}

\begin{proof}
The proof is by induction on $e$.  The condition on $H_s$ and $H_c$ is
necessary because $H_s$ maps variables to expressions in $E_o$, while
$H_c$ maps variables to real numbers.  Note that $\Sigma$ is unbound;
this is because, while $\Sigma$ is necessary for sampling
distributions in $E_o$, expressions in $E$ do not involve sampling.
We examine each of five cases individually.

\begin{itemize}
\item[$e_1 + e_2$] Let $(H_s, e_1) \judge_s \{ x_1 \}$ and $(H_s, e_2)
  \judge_s \{ x_2 \}$. Also let $(H_c, e_1) \judge_c v_1$ and $(H_c,
  e_2) \judge_c v_2$, so that $(H_c, e_1 + e_2) \judge_c v_1 + v_2 =
  v$.  By the definition of $\judge_s$, $(H_s, e_1 + e_2) \judge_s \{
  x_1 + x_2 \}$, and by induction $(\Sigma, x_1) \judge_o v_1$ and
  $(\Sigma, x_2) \judge_o v_2$.  Then by the definition of $\judge_o$,
  $(\Sigma, x) = (\Sigma, x_1 + x_2) \judge_o v_1 + v_2 = v$.  Thus
  this case is established.
\item[$e_1 * e_2$] Analogous to $e_1 + e_2$.
\item[$e_1 \div e_2$] Analogous to $e_1 + e_2$.
\item[$e_1 \div e_2$] Analogous to $e_1 + e_2$.
\item[$r$] $(H_s, r) \judge_s \{ r \}$ and $(\Sigma, r) \judge_c r$;
  on the other hand, $(H_c, r) \judge_c r$.  Thus this case is
  established.
\item[$v$] $(H_s, v) \judge_s H_s(v)$, while $(H_c, v) \judge_c
  H_c(v)$.  But by hypothesis, $(\Sigma, H_s(v)) \judge_e H_c(a)$, so
  this case is established.
\end{itemize}

These are all the cases present in the definition of $E$, so the lemma
is complete.

\end{proof}

\begin{lemma} \label{passert:lem:cond}
  For $c \in C$, let $(H_s, c) \judge_s \{ x \}$, and suppose that for
  every variable $a$, $(\Sigma, H_s(a)) \judge_e H_c(a)$.  Then $(H_c,
  c) \judge_c b$ if and only if $(\Sigma, x) \judge_o b$.
\end{lemma}

\begin{proof}
We again use induction, on $c$.  We examine each of five cases
individually.

\begin{itemize}
\item[$e_1 < e_2$] By the definition of $\judge_s$, $\{ x \} = \{ x_1
  + x_2 \}$.  Let $(H_c, e_1) \judge_c v_1$ and $(H_c, e_2) \judge_c
  v_2$, so that $b = [v_1 < v_2]$.  By lemma~\ref{passert:lem:expr}, $(\Sigma,
  x_1) \judge_o v_1$ and $(\Sigma, x_2) \judge_o v_2$, so $(\Sigma, x)
  \judge_o [v_1 < v_2] = b$.  Thus this case is established.
\item[$e_1 = e_2$] Analogous to $e_1 < e_2$.
\item[$c_1 \wedge c_2$] Let $(H_s, c_1) \judge_s \{ x_1 \}$ and $(H_s,
  c_2) \judge_s \{ x_2 \}$. Also let $(H_c, c_1) \judge_c b_1$ and
  $(H_c, c_2) \judge_c b_2$, so that $(H_c, c_1 \wedge c_2) \judge_c
  b_1 \wedge b_2 = v$.  By the definition of $\judge_s$, $(H_s, c_1
  \wedge c_2) \judge_s \{ x_1 \wedge x_2 \}$, and by induction
  $(\Sigma, x_1) \judge_o b_1$ and $(\Sigma, x_2) \judge_o b_2$.  Then
  by the definition of $\judge_o$, $(\Sigma, x) = (\Sigma, x_1 \wedge
  x_2) \judge_o b_1 + b_2 = b$.  Thus this case is established.
\item[$c_1 \vee c_2$] Analogous to $c_1 \wedge c_2$.
\item[$\neg c_1$] Let $(H_s, c_1) \judge_s \{ x_1 \}$ and $(H_c, c_1)
  \judge_c b_1$, so that $(H_c, \neg c_1) \judge_c \neg b_1$.  By the
  definition of $\judge_s$, $(H_s, \neg c_1) \judge_s \{ \neg x_1 \}$,
  and by induction $(\Sigma, x_1) \judge_o b_1$, so that $(\Sigma, x)
  \judge_o \neg b_1 = b$.  Thus this case is established.
\end{itemize}

These are all the cases present in the definition of $C$, so the lemma
is complete.

\end{proof}

We now prove a lemma which establishes equivalence for statements that
do not contain \textbf{while} loops.

\begin{lemma}
  \label{passert:lem:stmt}
  Let
  $(n, H_s, s) \to_s (m, H_s', s')$,
  where $s$ contains no \textbf{while} statements.
  Also suppose that $(\Sigma, n) \judge_o l$
  and $(\Sigma, m) \judge_o l + k$.
  Furthermore let $H_c$ be such that
  $(\Sigma, H_s(v)) \judge_o H_c(v)$
  for all variables $v$. Then
  $(\Sigma, H_c, s) \to_c^{*} (\Sigma', H_c', s')$,
  where
  $\Sigma = \sigma_1:\sigma_2:\dots:\sigma_k:\Sigma'$.
  Furthermore,
  $(\Sigma, H_s'(v)) \judge_o H_c'(v)$
  for all $v$.
\end{lemma}

\begin{proof}
  A few difficulties arise when attempting a naive induction:

  \begin{itemize}
  \item While $\judge_c$ and $\to_c$ consume an element of $\Sigma$,
    $\judge_s$ and $\to_s$ simply increment an offset. Our induction
    must show that this offset is correctly handled.
  \item While $\to_c$ only evaluates one side of an \textbf{if}
    statement, $\to_s$ evaluates both.  Proving that this is sound
    requires proving that the ``merge'' function correctly unifies the
    two branches.
  \item Non-terminating while loops, especially those involving
    sampling, are difficult to handle in the induction.  The statement
    of the lemma guarantees that the while loop must terminate (since
    $\to_s^{*}$ requires a finite number of steps), but the possibility
    for while loops to not terminate still complicates the proof.
  \end{itemize}
   
  The first problem is avoided by the statement of the lemma: we
  require that the symbolic semantics increment the sequence offset by
  exactly as many elements as the concrete semantics consumes.  The
  second problem requires a careful analysis of the ``merge''
  function.  This is also why we assume a single step in $\to_s$ but a
  number of steps in $\to_c^{*}$.  Finally, the last problem is avoided
  by a nested induction over the number of times the \textbf{while}
  loop is unrolled.  Since we assume the symbolic semantics terminate,
  the loop must at some point unroll fully, so the induction is
  founded.

  As mentioned, we induct over the number of steps taken by $\to_s^{*}$.
  At each step, we assume that the future steps will satisfy the
  statement of the lemma.  We consider each case individually.

  \begin{itemize}
  \item[$v := e$] Assume that $(H_c, e) \judge_c x_c$, so that
    $(\Sigma, H_c, v := e) \to_c (\Sigma, (v \mapsto x):H_c, \mathbf{skip})$.
    Furthermore, suppose $(H_s, e) \judge_s \{ x_s \}$, so that
    $(n, H_s, v := e) \to_s (n, (v \mapsto x_s):H_s, \mathbf{skip})$.
    By lemma~\ref{passert:lem:expr}, $(\Sigma, x_s) \judge_o x_s$.  But then,
    for all variables $v$, we have
    $(\Sigma, ((v \mapsto x_s):H_s)(v')) \judge_o ((v \mapsto x_c):H_c)(v')$
    for all $v'$.  If we set $\Sigma' = \Sigma$ and $k = 0$, we find
    that in this case our theorem is proven.
  \item[$v \leftarrow d$] Let $\Sigma = \sigma:\Sigma'$.  Then
    $(\Sigma, H_c, v \leftarrow d) \to_c (\Sigma', (v \mapsto d(\sigma)):H_c, \mathbf{skip})$
    On the other hand, in the symbolic semantics,
    $(\{ n \}, H_s, v \leftarrow d) \to_s (\{ n+1 \}, (v \mapsto \langle d, n \rangle):H_s, \mathbf{skip})$
    We can see that
    if $(\Sigma, \{ n \}) \judge_o l$, then $(\Sigma, \{n + 1\}) \judge_o l + 1$,
    forcing $k = 1$. Indeed, $\Sigma = \sigma_1:\Sigma'$.  Furthermore, since
    $(\Sigma, \langle d, n \rangle) \judge_o d(\sigma_1) = d(\sigma)$,
    we know that for all $v'$,
    $(\Sigma, ((v \mapsto \langle d, n \rangle):H_s)(v')) \judge_o ((v \mapsto d(\sigma)):H_c)(v')$.
    So this case is established.
  \item[\textbf{skip}] Since there are no symbolic steps
    for \textbf{skip}, the lemma is vacuously true.
  \item[$s_1 ; s_2$] This statement has two cases: where $s_1$ is
    \textbf{skip}, and where it is not.  If $s_1$ is \textbf{skip},
    the case is trivial, so suppose $s_1$ is not \textbf{skip}.
    Furthermore, let $(n, H_s, s_1) \to_s (m', H_s', s_1')$.
    By induction, we also have
    $(\Sigma, H_c, s_1) \to_c^{*} (\Sigma'', H_c', s_1')$,
    with the expected properties relating $\Sigma'$ and $k$, and
    $H_c'$ and $H_s'$.  But then since
    $(n, H_s, s_1 ; s_2) \to_s (m', H_s', s_1' ; s_2)$
    and
    $(\Sigma, H_c, s_1 ; s_2) \to_c^{*} (\Sigma'', H_c', s_1' ; s_2)$,
    this case is established with $m = m'$ and $\Sigma' = \Sigma''$.
    (We omit the lemma that
    $s_1 \to^{*} s_1'$ implies $s_1 ; s_2 \to^{*} s_1' ; s_2$,
    with the expected behavior of the other parameters.)
  \item[$\mathbf{if}\:c\:s_1\:s_2$]
    First, consider that per lemma~\ref{passert:lem:cond},
      we know that if $(H_c, c) \judge_c b$,
      and $(H_s, c) \judge_s \{ x_s \}$,
      then $(\Sigma, x_s) \judge_o b$.
    Now consider two sub-cases:
      $b$ is true, and $b$ is false.
    If $b$ is true,
      then for all expressions $y_t$ and $y_f$,
      the expression $\mathbf{if}\:x_s\:y_t\:y_f$
      must evaluate to the same result as $y_t$;
      otherwise if $b$ is false,
      to the same result as $y_f$.

    Now, depending on $b$,
    either
    $(\Sigma, H_c, \mathbf{if}\:c\:s_1\:s_2) \to (\Sigma', H_c', s_1)$
    or
    $(\Sigma, H_c, \mathbf{if}\:c\:s_1\:s_2) \to (\Sigma', H_c', s_2)$.
    We know that
    $(n, H_s, s_1) \to_s^{*} (m_t, H_{st}, \mathbf{skip})$
    and
    $(n, H_s, s_2) \to_s^{*} (m_f, H_{sf}, \mathbf{skip})$.
    But then by induction, we know that
    $(\Sigma, H_c, s_1) \to_c^{*} (\Sigma_t, H_{ct}, \mathbf{skip})$
    or
    $(\Sigma, H_c, s_2) \to_c^{*} (\Sigma_f, H_{cf}, \mathbf{skip})$,
    where the relationship of $\Sigma_t$ to $m_t$, of $\Sigma_f$ to
    $m_f$, of $H_{ct}$ to $H_{st}$, and of $H_{cf}$ to $H_{sf}$ are
    as expected.
    Thus, when
    $(n, H_s, \mathbf{if}\:c\:s_1\:s_2) \to (m, H_s', \mathbf{skip})$,
    we know that
    $(\Sigma, H_c, \mathbf{if}\:c\:s_1\:s_2) \to (\Sigma', H_c', \mathbf{skip})$
    as required, where $\Sigma'$ is $\Sigma_t$ or $\Sigma_f$ depending
    on the condition, and where $H_c'$ is $H_{ct}$ or $H_{cf}$, again
    depending on the loop condition.

    All that remains to prove is that the symbolic inference rule for
    the \textbf{if} rule correctly combines $H_{st}$ and $H_{sf}$, and
    likewise correctly combines $m_t$ and $m_f$.
    Recall that $b$ is the value of the loop condition, and the loop
    conditional evaluates symbolically to $x_s$.
    We do a case analysis on $b$.
    First, suppose $b$ is true.  Then $\Sigma' = \Sigma_t$, so we know
    that $\Sigma' = \sigma_1:\dotsb\sigma_k:\Sigma'$ where
    $(\Sigma, m) = (\Sigma, \mathbf{if}\:x_s\:m_t\:m_f) \judge_o k$.
    Similarly, since $H_s' = \mathtt{merge}(H_{sf}, H_{st}, x_s)$,
    we know that for all variable $v$,
    $(\Sigma, H_s'(v)) = (\Sigma, \mathtt{merge}(H_{st}, H_{sf}, x_s)(v))$
    This is equal to either $(\Sigma, \mathbf{if}\:x_s\:(H_{st}(v))\:(H_{sf}(v)))$
    or $(\Sigma, H_{st}(v))$, both of which evaluate to $H_{ct}(v) = H_c'(v)$
    because $x_s$ evaluates to $b$ which is true, and because
    $(\Sigma, H_{st}(v)) = H_{ct}(v)$ by induction.
    The case where $b$ is false is analogous to the case where $b$ is
    true.

    Thus this case is established.
  \end{itemize}

  This was the last remaining case (we assume that $s_1$ contains no
  \textbf{while} statements), so the lemma is done.
\end{proof}

We now extend the equivalence to programs that contain while loops.
We require that the symbolic evaluation terminate.

\begin{lemma} \label{passert:lem:whil}
    Let $(n, H_s, s) \to_s^{*} (m, H_s', \mathbf{skip})$.
  Further suppose that for all variables $v$,
  $(\Sigma, H_s(v)) \judge_o H_c(v)$.
  Then $(\Sigma, H_c, s) \to_c^{*} (\Sigma', H_c', \mathbf{skip})$,
  and furthermore for all variables $v$,
  $(\Sigma, H_s'(v)) = H_c'(v)$ and also
  $\Sigma = \sigma_1:\dotsb:\sigma_k:\Sigma'$,
  where $(\Sigma, m) \judge_o l + k$
  (where $(\Sigma, n) \judge_o l$).
\end{lemma}

\begin{proof}
  We proceed by structural induction on $s$.

  \begin{itemize}
  \item[$v := e$] There are no \textbf{while} loops in this statement,
    so it follows from lemma~\ref{passert:lem:stmt}.
  \item[$v \leftarrow d$] Analogous to $v := e$.
  \item[\textbf{skip}] Analogous to $v := e$.
  \item[$s_1 ; s_2$] 
    We must have
    $(n, H_s, s_1) \to_s^{*} (n', H_s'', \mathbf{skip})$,
    so by induction
    $(\Sigma, H_c, s_1) \to_c^{*} (\Sigma'', H_c'', \mathbf{skip})$,
    with the usual relation between $\Sigma''$ and $n'$, and between
    $H_s''$ and $H_c''$.
    But then again by induction
    $(n', H_s'', s_2) \to_s^{*} (m, H_s', \mathbf{skip})$
    implies
    $(\Sigma'', H_c'', s_2) \to_c^{*} (\Sigma', H_c', \mathbf{skip})$.
    Thus,
    $(\Sigma, H_c, s_1;s_2) \to_c^{*} (\Sigma', H_c', \mathbf{skip})$,
    and this case is established.
  \item[$\mathbf{if}\:c\:s_1\:s_2$]
      If $(n, H_s, \mathbf{if}\:c\:s_1\:s_2) \to^{*} (n', H_s', \mathbf{skip})$,
      we must have $(n, H_s, s_1) \to^{*} (n_t, H_{st}', \mathbf{skip})$
      and $(n, H_s, s_2) \to^{*} (n_f, H_{sf}', \mathbf{skip})$.  Then,
    analogously to the argument in lemma~\ref{passert:lem:stmt}, this case can
    be established.
  \item[$\mathbf{while}\:c\:s$]
    There are two inference rules concerning the symbolic semantics of
    \textbf{while} loops, so we must prove that both are sound.

    First consider the rule \textsc{While0}. If it applies, we must
    have $(\Sigma, x) \judge_o \mathbf{false}$
    for $(H_s, c) \judge_s \{ x \}$,
    and thus (by lemma~\ref{passert:lem:cond})
    $(H_c, c) \judge_c \mathbf{false}$.
    Then
    $(\Sigma, H_c, \mathbf{while}\:c\:s) \to^{*} (\Sigma, H_c, \mathbf{skip})$.
    But by assumption,
    $(n, H_s, \mathbf{while}\:c\:s) \to (n, H_s, \mathbf{skip})$,
    so the inductive statement holds.

    Second, consider the rule \textsc{While} in the symbolic
    semantics.  It is identical to the corresponding rule for
    \textbf{while}, so by induction this case is established.
  \end{itemize}

  These are all the cases for $S$, so the lemma is proven.
  
\end{proof}

Finally, we can prove our Theorem~\ref{passert:thm:main}.

\begin{theorem}
Let $(0, H_0, p) \judge_s \{ x \}$, where $x$ is a finite program.
Then $(\Sigma, H_0, p) \judge_c b$ if and only if $(\Sigma, x) \judge_o b$.
\end{theorem}

\begin{proof}
  First, note that $H_c = H_s = H_0$, so that
  $(\Sigma, H_s(v)) \judge_o H_c(v)$ for all $v$
  (by the rule for constants).

  Let the program $p$ be $s\:;;\:\passert\:c$.
  If $(0, H_0, p) \judge_s \{ x \}$, then
  $(0, H_0, s) \to_s^{*} (n, H_s, \mathbf{skip})$.
  Then by lemma~\ref{passert:lem:whil},
  $(\Sigma, H_0, s) \to_s^{*} (\Sigma', H_c, \mathbf{skip}$,
  with the expected relation between $H_c$ and $H_s$.
  But then due to this relation,
  if $(H_s, c) \judge_s \{ y \}$,
  $(H_c, c) \judge_c b$ if and only if $(\Sigma, y) \judge_o b$
  (the lemma to prove this would be a straightforward induction over $y$).

  Thus, $(\Sigma, H_0, p) \judge_c b$ if
  and only if $(\Sigma, x) \judge_o b$,
  and our theorem is proven.
\end{proof}

While complex, this theorem shows that the distribution extraction
performed by \tool is sound.
