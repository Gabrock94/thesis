\section{Discussion}

Probabilistic assertions express quality constraints, not only for
approximate programming but for any computational domain that uses randomness
to do its work.
In contrast to the other quality-focused work in this dissertation,
the probabilistic assertion verification workflow in this chapter makes the
closest connections to traditional statistical reasoning.
It is also the most general approach:
the techniques
applies to ``probabilistic programming languages'' as defined
by Kozen~\cite{kozen}: ordinary languages extended with random calls.
In exchange for its generality, the approach makes weaker guarantees than, for
example, \chref{decaf}'s conservative probability bounds:
the basis in sampling always leaves room for false positives.
A healthy ecosystem for approximate programming will need techniques from
across the strength--generality continuum.
