\abstract{%
  % toddm: inference as example for ML

  Traditional assertions express correctness properties that must hold
  on every program execution.  However, many applications have
  probabilistic outcomes and consequently their correctness properties
  are also probabilistic (e.g., they identify faces in images, consume
  sensor data, or run on unreliable hardware). Traditional
  assertions do not capture these correctness
  properties. % and current verification tools prove
  % properties that hold for \emph{every} execution
  % For programs with
  % probabilistic outcomes, we
  This paper proposes that programmers express probabilistic
  correctness properties with \emph{probabilistic assertions} and
  describes a new \emph{probabilistic evaluation} approach to
  efficiently verify these assertions.
 %  To which this paper proposes \emph{probabilistic assertions} and
 %  \emph{probabilistic evaluation}. % to efficiently verify them.
  Probabilistic assertions are Boolean expressions that express the
  \emph{probability} that a property will be true in a given execution
  rather than asserting that the property must always be true.
  Given either specific inputs or distributions on the input space,
  probabilistic evaluation verifies probabilistic assertions
  by first performing \emph{distribution
    extraction} to represent the program
  as a Bayesian network. Probabilistic evaluation then uses statistical
  properties to simplify
  this representation to efficiently compute assertion
  probabilities directly or with sampling.  Our approach is a mix of
  both static and dynamic analysis: distribution extraction
  statically
  builds and optimizes the Bayesian network representation and
  sampling dynamically interprets this representation.
  We implement our approach in a
  tool called \tool for C and C++ programs. We evaluate
  expressiveness, correctness, and performance of \tool on programs
  that use sensors, perform approximate computation, and obfuscate
  data for privacy.  Our case studies demonstrate that probabilistic
  assertions describe useful correctness properties and that \tool
  efficiently verifies them.

  % We evaluate an implementation of the analysis using C and C++ programs
  % that use sensors, perform approximate computation, and obfuscate
  % data for privacy.
  % We demonstrate that probabilistic assertions express useful properties of
  % these programs that are not captured by traditional assertions. 
  % And, our analysis evaluates probabilistic assertions \xxx{?} faster on average
  % than a naive checker.
}

%%% Local Variables: 
%%% mode: latex
%%% TeX-master: "paper"
%%% End: 
