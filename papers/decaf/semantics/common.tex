\usepackage{amsfonts}
\usepackage{amsmath}
\usepackage{mathpartir}
\usepackage{xspace}
\usepackage{amsthm}
\usepackage{color}
\usepackage{chngcntr}

\newcommand{\TODO}[1]{\textcolor{red}{\textsf{\textbf{#1}}}}

% Code-like type that works in math mode (as well as text mode).
\newcommand{\mcode}[1]{\text{\normalfont{\sffamily\small{#1}}}}

% Semantics-y macros.
\newcommand{\defeq}{::=}
\newcommand{\judge}{\Downarrow}
\newcommand{\alt}{\:|\:}
\newcommand{\skips}{\mathbf{skip}}
%\newcommand{\prarrow}[1]{\xrightarrow{#1}}
%\newcommand{\rarrow}{\prarrow{1.0}}
\newcommand{\prarrow}[1]{\longrightarrow}
\newcommand{\rarrow}{\longrightarrow}
\newcommand{\pjudge}[1]{\judge_{#1}}
\newcommand{\cjudge}{\pjudge{1.0}}
\newcommand{\csemi}{\mcode{;}} % Code semicolon

% amsthm
\newtheorem{theorem}{Theorem}
\newtheorem{definition}{Definition}
\newtheorem{lemma}{Lemma}

% Reset the theorem counters so we repeat numbers when restating the same
% definitions/theorems in the main body vs. the appendix.
\counterwithin*{theorem}{section}
\counterwithin*{definition}{section}
\counterwithin*{lemma}{section}

% Qualifier shorthand.
\newcommand{\Approx}{\text{\normalfont{\sffamily\small{@Approx}}}}
\newcommand{\Precise}{\text{\normalfont{\sffamily\small{@Precise}}}}
\newcommand{\Dyn}{\text{\normalfont{\sffamily\small{@Dyn}}}}
\newcommand{\ApproxN}[1]{\Approx(#1)}

% Math macros for probability stuff.
\newcommand{\prob}[1]{\ensuremath{\Pr\!\left[#1\right]}}
\newcommand{\expc}[1]{\ensuremath{\mathrm{E}\!\left[#1\right]}}
\newcommand{\varc}[1]{\ensuremath{\mathrm{Var}\!\left[#1\right]}}

% Describing the project.
\newcommand{\lang}{DECAF\xspace}
\newcommand{\bench}[1]{\textsf{#1}}
