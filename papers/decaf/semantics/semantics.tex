\label{decaf:app:semantics}

This appendix expands on the formalism for DECAF, the probability-types
language in \chref{decaf}.
We present the full syntax, static semantics, and dynamic semantics for the
core \lang language.
We prove a \emph{soundness} theorem that embodies the probability type
system's fundamental accuracy guarantee.
This appendix corresponds to the appendix for the main DECAF paper in OOPSLA
2015~\cite{decaf}.

\section{Syntax}

We formalize a core of \lang without inference.
The syntax for statements, expressions, and types is:
%
\begin{align*}
    s &\defeq
        T \> v := e \alt
        v := e \alt
        s\:\csemi\:s \alt
        \mathbf{if}\:e\:s\:s \alt
        \mathbf{while}\:e\:s \alt
        \mathbf{skip} \\
    e &\defeq
        c \alt
        v \alt
        e \oplus_p e \alt
        \mcode{endorse}(p, e) \alt
        \mcode{check}(p, e) \alt
        \mcode{track}(p, e) \\
    \oplus &\defeq
        + \alt - \alt \times \alt \div \\
    T &\defeq
        q \> \tau \\
    q &\defeq
        \ApproxN{p} \alt
        \Dyn \\
    \tau &\defeq
        \mcode{int} \alt \mcode{float} \\
    v &\in \mathrm{variables},
        ~c \in \mathrm{constants},
        ~p \in [0.0, 1.0]
\end{align*}

%
For the purpose of the static and dynamic semantics, we also define values $V$, heaps
$H$, dynamic probability maps $D$, true probability maps $S$, and static
contexts $\Gamma$:
%
\begin{align*}
    V &\defeq
        c \alt
        \Box \\
    H &\defeq
        \cdot \alt
        H, v \mapsto V \\
    D &\defeq
        \cdot \alt
        D, v \mapsto p \\
    S &\defeq
        \cdot \alt
        S, v \mapsto p \\
    \Gamma &\defeq
        \cdot \alt
        \Gamma, v \mapsto T
\end{align*}
%
We define $H(v)$, $D(v)$, $S(v)$, and $\Gamma(v)$ to denote variable lookup in
these maps.


\section{Typing}

The type system defines the static semantics for the core language. We first
give typing judgments first for expressions and then for statements.

\subsection{Operator Typing}

We introduce a helper ``function'' that determines the unqualified result type
of a binary arithmetic operator.

\vspace{1ex}\noindent
$\boxed{\text{optype}(\tau_1, \tau_2) = \tau_3}$
%
\begin{mathpar}
    \text{optype}(\tau,\tau) = \tau

    \text{optype}(\mcode{int},\mcode{float}) =
           \mcode{float}

    \text{optype}(\mcode{float},\mcode{int}) =
           \mcode{float}
\end{mathpar}
%
Now we can give the types of the binary operator expressions themselves.
There are two cases: one for statically-typed operators and one for
dynamic tracking.
The operands may not mix static and dynamic qualifiers (recall that the
compiler inserts \mcode{track} casts to introduce dynamic tracking when
necessary).

\vspace{1ex}\noindent
$\boxed{\Gamma \vdash e : T}$
%
\begin{mathpar}
    \inferrule[op-static-types]
    {\Gamma \vdash e_1 : \ApproxN{p_1} \> \tau_1 \\
    \Gamma \vdash e_2 : \ApproxN{p_2} \> \tau_2 \\
    \tau_3 = \text{optype}(\tau_1, \tau_2) \\
    p' = p_1 \cdot p_2 \cdot p_\text{op}}
    {\Gamma \vdash e_1 \oplus_{p_\text{op}} e_2 : \ApproxN{p'} \> \tau_3}

    \inferrule[op-dyn-types]
    {\Gamma \vdash e_1 : \Dyn \> \tau_1 \\
    \Gamma \vdash e_2 : \Dyn \> \tau_2 \\
    \tau_3 = \text{optype}(\tau_1, \tau_2)}
    {\Gamma \vdash e_1 \oplus_p e_2 : \Dyn \> \tau_3}
\end{mathpar}
%
In the static case, the output probability is the product of the probabilities
for the left-hand operand, right-hand operand, and the operator itself.
Section~\ref{decaf:sec:types} gives the probabilistic intuition behind this rule.

\subsection{Other Expressions}

The rules for constants and variables are straightforward. Literals are given
the precise ($p = 1.0$) type.
%
\begin{mathpar}
    \inferrule[const-int-types]
    {c \text{ is an integer}}
    {\Gamma \vdash c : \ApproxN{1.0} \> \mcode{int}}

    \inferrule[const-float-types]
    {c \text{ is not an integer}}
    {\Gamma \vdash c : \ApproxN{1.0} \> \mcode{float}}

    \inferrule[var-types]
    {T = \Gamma(v)}
    {\Gamma \vdash v : T}
\end{mathpar}

Endorsements, both checked and unchecked, produce the explicitly requested type.
(Note that \mcode{check} is sound but \mcode{endorse} is potentially unsound:
our main soundness theorem, at the end of this appendix, will exclude the
latter from the language.)
Similarly, \mcode{track} casts produce a dynamically-tracked type given a
statically-tracked counterpart.
%
\begin{mathpar}
    \inferrule[endorse-types]
    {\Gamma \vdash e : q \> \tau}
    {\Gamma \vdash \mcode{endorse}(p,e) : \ApproxN{p} \> \tau}

    \inferrule[check-types]
    {\Gamma \vdash e : \Dyn \> \tau}
    {\Gamma \vdash \mcode{check}(p,e) : \ApproxN{p} \> \tau}

    \inferrule[track-types]
    {\Gamma \vdash e : \ApproxN{p'} \> \tau \\
    p \le p'}
    {\Gamma \vdash \mcode{track}(p, e) : \Dyn \> \tau}
\end{mathpar}


\subsection{Qualifiers and Subtyping}

A simple subtyping relation, introduced in Section~\ref{decaf:sec:types}, makes high-probability types subtypes of their
low-probability counterparts.

\vspace{1ex}\noindent
$\boxed{T_1 \prec T_2}$
\begin{mathpar}
    \inferrule[subtyping]
    {p \geq p'}
    {\ApproxN{p} \> \tau
     \prec \ApproxN{p'} \> \tau}
\end{mathpar}
%
Subtyping uses a standard subsumption rule.
%
\begin{mathpar}
    \inferrule[subsumption]
    {T_1 \prec T_2 \\ \Gamma \vdash e : T_1}
    {\Gamma \vdash e : T_2}
\end{mathpar}


\subsection{Statement Typing}

Our typing judgment for statements builds up the context $\Gamma$.

\vspace{1ex}\noindent
$\boxed{\Gamma_1 \vdash s : \Gamma_2}$
\begin{mathpar}
    \inferrule[skip-types]
    { }
    {\Gamma \vdash \mcode{skip} : \Gamma}

    \inferrule[seq-types]
    {\Gamma_1 \vdash s_1 : \Gamma_2 \\
    \Gamma_2 \vdash s_2 : \Gamma_3}
    {\Gamma_1 \vdash s_1 ; s_2 : \Gamma_3}

    \inferrule[decl-types]
    {\Gamma \vdash e : T \\
    v \notin \Gamma}
    {\Gamma \vdash T \> v := e
    : \Gamma, v : T}

    \inferrule[mutate-types]
    {\Gamma \vdash e : T \\
    \Gamma(v) = T}
    {\Gamma \vdash v := e
    : \Gamma}

    \inferrule[if-types]
    {\Gamma \vdash e : \ApproxN{1.0} \> \tau \\
    \Gamma \vdash s_1 : \Gamma_1 \\
    \Gamma \vdash s_2 : \Gamma_2}
    {\Gamma \vdash \mathbf{if}\:e\:s_1\:s_2 : \Gamma}

    \inferrule[while-types]
    {\Gamma \vdash e : \ApproxN{1.0} \> \tau \\
    \Gamma \vdash s : \Gamma'}
    {\Gamma \vdash \mathbf{while}\:e\:s : \Gamma}
\end{mathpar}
%
The conditions in \mcode{if} and \mcode{while} statements are required to have
the precise type ($p = 1.0$).


\section{Operational Semantics}

We use a large-step operational semantics for expressions and small-step
semantics for statements.
Both are nondeterministic: values produced by approximate operators can
produce either an error value $\Box$ or a concrete number.

\subsection{Expression Semantics}

There are two judgments for expressions: one for statically typed
expressions and one where dynamic tracking is used.
The former, $H ; D ; S ;  e \pjudge{p} V$, indicates that the expression
$e$ produces a value $V$, which is either a
constant $c$ or the error value $\Box$, and $p$ is the
probability that $V \ne \Box$.
The latter judgment, $H ; D ; S ; e \pjudge{p} V, p_d$, models
dynamically-tracked expression evaluation. In addition to a value $V$, it also produces a computed
probability value $p_d$ reflecting the compiler's conservative bound on
the reliability of $e$'s value.
That is, $p$ is the ``true'' probability that $V \ne \Box$ whereas
$p_d$ is the dynamically computed conservative bound for $p$.

In these judgments, $H$ is the heap mapping variables to values and $D$ is the {dynamic
probability map for \mcode{@Dyn}-typed variables maintained by the compiler.
The $S$ probability map is used for our type soundness proof: it maintains the
actual probability that a variable is correct.

\paragraph{Constants}
Literals are always tracked statically.
\begin{mathpar}
    \inferrule[const]
    { }
    {H ; D ; S ; c \cjudge c}
\end{mathpar}

\paragraph{Variables}
Variable lookup is dynamically tracked when the variable is present in the
tracking map $D$.
The probability $S(v)$ is the chance that the variable does not hold $\Box$.
\begin{mathpar}
    \inferrule[var]
    {v \not\in D}
    {H ; D ; S ; v \pjudge{S(v)} H(v)}

    \inferrule[var-dyn]
    {v \in D}
    {H ; D ; S ; v \pjudge{S(v)} H(v), D(v)}
\end{mathpar}

\paragraph{Endorsements}
Unchecked (unsound) endorsements apply only to statically-tracked values and
do not affect the correctness probability.
\begin{mathpar}
    \inferrule[endorse]
    {H ; D ; S ; e \pjudge{p} V}
    {H ; D ; S ; \mcode{endorse}(p_e,e) \pjudge{p} V}
\end{mathpar}

\paragraph{Checked Endorsements}
Checked endorsements apply to dynamically-tracked values and produce
statically-tracked values.
The tracked probability must meet or exceed the check's required probability;
otherwise, evaluation gets stuck.
(Our implementation throws an exception.)
\begin{mathpar}
    \inferrule[check]
    {H ; D ; S ; e \pjudge{p} V, p_1 \\
    p_1 \geq p_2}
    {H ; D ; S ; \mcode{check}(p_2,e) \pjudge{p} V}
\end{mathpar}

\paragraph{Tracking}
The static-to-dynamic cast expression allows statically-typed values to be
combined with dynamically-tracked ones.
The tracked probability field for the value is initialized to match the
explicit probability in the expression.
%
\begin{mathpar}
    \inferrule[track]
    {H ; D ; S ; e \pjudge{p} V}
    {H ; D ; S ; \mcode{track}(p_d, e) \pjudge{p} V, p_d}
\end{mathpar}

\paragraph{Operators}
Binary operators can be either statically tracked or dynamically tracked.
In each case, either operand can be the error value or a constant.
When either operand is $\Box$, the result is $\Box$.
When both operands are non-errors, the operator itself can
(nondeterministically) produce either $\Box$ or a correct result.
The correctness probability, however, is the same for all three rules:
intuitively, the probability itself is deterministic even though the semantics
overall are nondeterministic.

In these rules, $c_1 \oplus c_2$ without a probability subscript denotes the appropriate binary
operation on integer or floating-point values.
The statically-tracked cases are:
\begin{mathpar}
    \inferrule[op]
    {H ; D ; S ;  e_1 \pjudge{p_1} c_1 \\
    H ; D ; S ; e_2 \pjudge{p_2} c_2 \\
    p = p_1 \cdot p_2 \cdot p_\text{op}}
    {H ; D ; S ; e_1 \oplus_{p_\text{op}} e_2 \pjudge{p} c_1 \oplus c_2}

    \inferrule[op-operator-incorrect]
    {H ; D ; S ; e_1 \pjudge{p_1} c_1 \\
    H ; D ; S ; e_2 \pjudge{p_2} c_2 \\
    p = p_1 \cdot p_2 \cdot p_\text{op}}
    {H ; D ; S ; e_1 \oplus_{p_\text{op}} e_2 \pjudge{p} \Box}

    \inferrule[op-operands-incorrect]
    {H ; D ; S ; e_1 \pjudge{p_1} \Box \text{ or }
    H ; D ; S ; e_2 \pjudge{p_2} \Box \\
    p = p_1 \cdot p_2 \cdot p_\text{op}}
    {H ; D ; S ; e_1 \oplus_{p_\text{op}} e_2 \pjudge{p} \Box}
\end{mathpar}

The dynamic-tracking rules are similar, with the additional propagation of the
conservative probability field.
%
\begin{mathpar}
    \inferrule[op-dyn]
    {H ; D ; S ; e_1 \pjudge{p_1} c_1, p_{d1} \\
    H ; D ; S ; e_2 \pjudge{p_2} c_2, p_{d2} \\
    p = p_1 \cdot p_2 \cdot p_\text{op}}
    {H ; D ; S ; e_1 \oplus_{p_\text{op}} e_2 \pjudge{p} c_1 \oplus c_2, \;
    p_{d1} \cdot p_{d2} \cdot p_\text{op}}

    \inferrule[op-dyn-operator-incorrect]
    {H ; D ; S ; e_1 \pjudge{p_1} c_1, p_{d1} \\
    H ; D ; S ; e_2 \pjudge{p_2} c_2, p_{d2} \\
    p = p_1 \cdot p_2 \cdot p_\text{op}}
    {H ; D ; S ; e_1 \oplus_{p_\text{op}} e_2 \pjudge{p} \Box, \;
    p_{d1} \cdot p_{d2} \cdot p_\text{op}}

    \inferrule[op-dyn-operands-incorrect]
    {H ; D ; S; e_1 \pjudge{p_1} \Box, p_{d1} \text{ or }
    H ; D ; S ; e_2 \pjudge{p_2} \Box, p_{d2} \\
    p = p_1 \cdot p_2 \cdot p_\text{op}}
    {H ; D ; S ; e_1 \oplus_{p_\text{op}} e_2 \pjudge{p} \Box, \;
    p_{d1} \cdot p_{d2} \cdot p_\text{op}}
\end{mathpar}

\subsection{Statement Semantics}

The small-step judgment for statements is
$H ; D ; S ; s \prarrow{p} H' ; D' ; S' ; s'$.

\paragraph{Assignment}
The rules for assignment (initializing a fresh variable) take advantage of nondeterminism in the evaluation of
expressions to nondeterministically update the heap with either a constant or
the error value, $\Box$.

\vspace{1ex}\noindent
$\boxed{H ; D ; s \prarrow{p} H' ; D' ; s'}$
\begin{mathpar}
    \inferrule[assign]
    {H ; D ; S ; e \pjudge{p} V}
    {H ; D ; S ; \ApproxN{p'} \> \tau \> v := e
    \prarrow{p} \\\\
    H, v \mapsto V ; D ; S, v \mapsto p ; \skips}

    \inferrule[assign-dyn]
    {H ; D ; S ; e \pjudge{p} V, p_d}
    {H ; D ; S ; \Dyn \> \tau \> v := e
    \prarrow{p} \\\\
    H, v \mapsto V ; D, v \mapsto p_d ; S, v \mapsto p ; \skips}
\end{mathpar}

Mutation works like assignment, but existing variables are overwritten
in the heap.
\begin{mathpar}
    \inferrule[mutate]
    {H ; D ; S ; e \pjudge{p} V}
    {H ; D ; S ; v := e
    \prarrow{p}
    H, v \mapsto V ; D ; S, v \mapsto p ; \skips}

    \inferrule[mutate-dyn]
    {H ; D ; e \pjudge{p} V, p_d}
    {H ; D ; v := e
    \prarrow{p}
    H, v \mapsto V ; D, v \mapsto p_d ; S, v \mapsto p ; \skips}
\end{mathpar}

\paragraph{Sequencing}
Sequencing is standard and deterministic.
\begin{mathpar}
    \inferrule[seq-skip]
    { }
    {H ; D ; S ; \skips\csemi s \rarrow H ; D ; S ; s}

    \inferrule[seq]
    {H ; D ; S ; s_1 \prarrow{p} H' ; D' ; S' ; s_1'}
    {H ; D ; S ; s_1\csemi s_2 \prarrow{p} H' ; D' ; S' ; s_1'\csemi s_2}
\end{mathpar}

\paragraph{If and While}
The type system requires conditions in \mcode{if} and \mcode{while} control
flow decisions to be deterministic ($p = 1.0$).
\begin{mathpar}
    \inferrule[if-true]
    {H ; D ; S ; e \cjudge c \\
    c \neq 0}
    {H ; D ; S ; \mathbf{if}\:e\:s_1\:s_2 \rarrow H ; D ; S ; s_1}

    \inferrule[if-false]
    {H ; D ; S ; e \cjudge c \\
    c = 0}
    {H ; D ; S ; \mathbf{if}\:e\:s_1\:s_2 \rarrow H ; D ; S ; s_2}

    \inferrule[while]
    { }
    {H ; D ; S ; \mathbf{while}\:e\:s \rarrow
    H ; D ; S ; \mathbf{if}\:e\:(s\csemi \mathbf{while}\:e\:s)\: \skips}
\end{mathpar}

\section{Theorems}

The purpose of the formalism is to express a soundness theorem that shows that
\lang's probability types act as lower bounds on programs' run-time
probabilities.
We also sketch the proof of a theorem stating that the bookkeeping probability map, $S$,
is eraseable: it is used only for the purpose of our soundness theorem and
does not affect the heap.

\subsection{Soundness}

The soundness theorem for the language states that the probability types are
lower bounds on the run-time correctness probabilities.
Specifically, both the static types \ApproxN{p} and the dynamically tracked
probabilities in $D$ are lower bounds for the corresponding probabilities in
$S$.

To state the soundness theorem, we first define well-formed dynamic states.
We write
$\vdash D, S : \Gamma$
to denote that the dynamic probability field map $D$
and the actual probability map $S$
are \emph{well-formed} in the static context $\Gamma$.

\begin{definition}[Well-Formed]
$\vdash D, S : \Gamma$
iff
for all $v \in \Gamma$,
\begin{itemize}
\item
If $\Gamma(v) = \ApproxN{p} \> \tau$,
then $p \le S(v)$ or $v \notin S$.
\item
If $\Gamma(v) = \Dyn \> \tau$,
then $D(v) \le S(v)$ or $v \notin S$.
\end{itemize}
\end{definition}


We can now state and prove the soundness theorem. We first give the main
theorem and then two
preservation lemmas, one for expressions and one for statements.

\begin{theorem}[Soundness]
For all programs $s$ with no \mcode{endorse} expressions,
for all $n \in \mathbb{N}$
where $\cdot ; \cdot ; \cdot ; s
\rarrow^n
H ; D ; S ; s'$,
%
if $\cdot \vdash s : \Gamma$,
then
$\vdash D, S : \Gamma$.
\end{theorem}


\begin{proof}
Induct on the number of small steps, $n$.
When $n = 0$, both conditions hold trivially since $v \notin \cdot$ for all
$v$.

For the inductive case, we assume that:
$$\cdot ; \cdot ; \cdot ; s \rarrow^n
H_1 ; D_1; S_1; s_1$$
and:
$$H_1 ; D_1 ; S_1 ; s_1 \rarrow
H_2 ; D_2 ; S_2 ; s_2$$
and that
$\vdash D_1, S_1 : \Gamma$.
We need to show that
$\vdash D_2, S_2 : \Gamma$ also.
The Statement Preservation lemma, below, applies and meets this goal.
\end{proof}

The first lemma is a preservation property for expressions. We will use this
lemma to prove a corresponding preservation lemma for statements, which in turn applies to
prove the main theorem.

\begin{lemma}[Expression Preservation]
For all expressions $e$ with no \mcode{endorse} expressions
where $\Gamma \vdash e : T$
and where $\vdash D, S : \Gamma$,
\begin{itemize}
\item
If $T = \ApproxN{p} \> \tau$,
and $H ; D ; S ; e \pjudge{p'} V$,
then $p \le p'$.
\item
If $T = \Dyn \> \tau$,
and $H ; D ; S ; e \pjudge{p'} V, p$,
then $p \le p'$.
\end{itemize}
\end{lemma}

\begin{proof}
Induct on the typing judgment for expressions, $\Gamma \vdash e : T$.

\paragraph{Case \textsc{op-static-types}}
Here, $e = e_1 \oplus_{p_{op}} e_2$
and $T = \ApproxN{p} \> \tau$.
We also have types for the operands:
$\Gamma \vdash e_1 : \ApproxN{p_1} \> \tau_1$
and
$\Gamma \vdash e_2 : \ApproxN{p_2} \> \tau_2$.

By inversion on $H ; D ; S ; e \pjudge{p'} V$
(in any of the three operator cases \textsc{op}, \textsc{op-operator-incorrect}, or
\textsc{op-operands-incorrect}),
$p' = p_1' \cdot p_2' \cdot p_\text{op}$
where $H ; D ; S ; e_1 \pjudge{p_1'} V_1$
and $H ; D ; S ; e_2 \pjudge{p_2'} V_2$.

By applying the induction hypothesis to $e_1$ and $e_2$, we have
$p_1 \le p_1'$
and
$p_2 \le p_2'$.
Therefore,
$p_1 \cdot p_2 \cdot p_\text{op}
\le
p_1' \cdot p_2' \cdot p_\text{op}$
and, by substitution,
$p \le p'$.

\paragraph{Case \textsc{op-dyn-types}}
The case for dynamically-tracked expressions is similar.
Here, $e = e_1 \oplus_{p_{op}} e_2$ and $T = \Dyn \> \tau$,
and the operand types are
$\Gamma \vdash e_1 : \Dyn \> \tau_1$ and
$\Gamma \vdash e_2 : \Dyn \> \tau_2$.

By inversion on $H ; D ; S ; e \pjudge{p'} V, p$ (in any of the cases
\textsc{op-dyn, op-dyn-operator-incorrect}, or
\textsc{op-dyn-operands-incorrect}), $p' = p'_1 \cdot p'_2 \cdot p_\text{op}$,
$p = p_{d1} \cdot p_{d2} \cdot p_\text{op}$
where $H ; D ; S ; e_1 \pjudge{p'_1} V_1, p_{d1}$ and
$H ; D ; S ; e_2 \pjudge{p'_2} V_2, p_{d2}$.

By applying the induction hypothesis to $e_1$ and $e_2$, we have
$p_{d1} \le p'_1$ and $p_{d2} \le p'_2$.
Therefore,
$p_{d1} \cdot p_{d2} \cdot p_\text{op} \le p'_1 \cdot p'_2 \cdot p_\text{op}$
and, by substitution, $p \le p'$.

\paragraph{Cases \textsc{const-int-types} and \textsc{const-float-types}}
Here, we have that
$\Gamma \vdash e : \ApproxN{p} \> \tau$
where
$\tau \in \{\text{int, float}\}$
and
$p = 1.0$.

By inversion on $H ; D ; S ; e \pjudge{p'} V$ we get $p' = 1.0$.

Because $1.0 \le 1.0$, we have $p \le p'$.

\paragraph{Case \textsc{var-types}}
Here, $e = v$, $\Gamma \vdash v : T$.
Destructing $T$ yields two subcases.
    \begin{itemize}
    \item Case $T = \ApproxN{p} \> \tau$:
    By inversion on $H ; D ; S ; e \pjudge{p'} V$ we have $p' = S(V)$.

    The definition of well-formedness gives us $p \le S(V)$.

    By substitution, $p \le p'$.

    \item Case $T = \Dyn \> \tau$:
    By inversion on $H ; D ; S ; e \pjudge{p'} V, p$, we have $p' = S(V)$ and
    $p = D(V)$.

    Well-formedness gives us $D(V) \le S(V)$.

    By substitution, $p \le p'$.
    \end{itemize}

\paragraph{Case \textsc{endorse-types}}
The expression
$e$ may not contain \mcode{endorse} expressions so the claim hold vacuously.

\paragraph{Case \textsc{check-types}}
Here, $e = \mcode{check}(p, e_c)$.

By inversion on $H ; D ; S ; e \pjudge{p'} V$, we have
$H ; D ; S ; e_c \pjudge{p'} V, p''$, and
$p \le p''$.

By applying the induction hypothesis to
$H ; D ; S ; e_c \pjudge{p'} V, p''$, we get $p'' \le p'$.

By transitivity of inequalities, $p \le p'$.

\paragraph{Case \textsc{track-types}}
Here, $e = \mcode{track}(p_t, e_t)$,
$\Gamma \vdash e_t : \ApproxN{p''}$,
and $p \le p''$.

By inversion on
$H ; D ; S ; e \pjudge{p'} V, p$, we get
$H ; D ; S ; e_t \pjudge{p'} V$.

By applying the induction hypothesis to
$H ; D ; S ; e_t \pjudge{p'} V$, we get $p'' \le p'$.

By transitivity of inequalities, $p \le p'$.

\paragraph{Case \textsc{subsumption}}
The case where
$T = \ApproxN{p} \> \tau$
applies.
There is one rule for subtyping, so we have
$\Gamma \vdash e : \ApproxN{p_s} \> \tau$
where
$p_s \ge p$.
By induction,
$p_s \le p'$,
so
$p \le p'$.
\end{proof}

Finally, we use this preservation lemma for expressions to prove a
preservation lemma for statements, completing the main soundness proof.

\begin{lemma}[Statement Preservation]
For all programs $s$ with no \mcode{endorse} expressions,
if $\Gamma \vdash s : \Gamma'$,
and $\vdash D , S : \Gamma$,
and $H ; D ; S \rarrow H' ; D' ; S'$,
then
$\vdash D' , S' : \Gamma'$.
\end{lemma}

\begin{proof}
We induct on the derivation of the statement typing judgment,
$\Gamma \vdash s : \Gamma'$.

\paragraph{Cases \textsc{skip-types}, \textsc{if-types}, and
\textsc{while-types}}
    In these cases, $\Gamma = \Gamma'$,
    $D = D'$,
    and $S = S'$,
    so preservation holds trivially.

\paragraph{Case \textsc{seq-types}}
    Here, $s = s_1 ; s_2$ and the typing judgments for the two component
    statements are
    $\Gamma \vdash s_1 : \Gamma_2$
    and
    $\Gamma_2 \vdash s_2 : \Gamma'$.
    If $s_1 = \skips$, then the case is trivial.
    Otherwise, by inversion on the small step,
    $H ; D ; S ; s_1 \rarrow
    H' ; D' ; S' ; s_1'$
    and, by the induction hypothesis,
    $\vdash D_1', S_1' : \Gamma$.

\paragraph{Case \textsc{decl-types}}
    The statement $s$ is $T v := e$
    where $\Gamma \vdash e : T$
    and $\Gamma' = \Gamma, v : T$.
    We consider two cases: either $T = \ApproxN{p} \> \tau$
    or $T = \Dyn \> \tau$.
    In either case, the expression preservation lemma applies.

    In the first case,
    $H ; D ; S ; e \pjudge{p'} V$ where $p \le p'$ via expression preservation
    and, by inversion,
    $S' = S, v \mapsto p$ and $D' = D$.
    Since $S'(v) = p \le p'$,
    the well-formedness property $\vdash D, S : \Gamma'$ continues to hold.

    In the second case
    $H ; D ; S ; e \pjudge{p'} V, p_d$ where $p_d \le p'$.
    By inversion,
    $S' = S, v \mapsto p$ and $D' = D, v \mapsto p_d$.
    Since $D'(v) = p_d \le p'$,
    we again have $\vdash D, S : \Gamma'$.

\paragraph{Case \textsc{mutate-types}}
    The case where $s$ is $v := e$ proceeds similarly to the above case for
    declarations.
\end{proof}


\subsection{Erasure of Probability Bookkeeping}

We state (and sketch a proof for) an \emph{erasure} property that
shows that the ``true'' probabilities in our semantics, called $S$, do not
affect execution.
This property emphasizes that $S$ is bookkeeping for the purpose of stating
our soundness result---it corresponds to no run-time data.
Intuitively, the theorem states that the steps taken in our dynamic semantics
are insensitive to $S$: that $S$ has no effect on which $H'$, $D'$, or $s'$
can be produced.

In this statement, $\mathrm{Dom}(S)$ denotes the set of variables in the mapping
$S$.

\begin{theorem}[Bookkeeping Erasure]
If
$H ; D ; S_1 ; s
\rarrow^n
H' ; D' ; S_1' ; s'$,
then for any probability map $S_2$
for which $\mathrm{Dom}(S_1) = \mathrm{Dom}(S_2)$,
there exists another map $S_2'$ such that
$H ; D ; S_2 ; s
\rarrow^n
H' ; D' ; S_2' ; s'$.
\end{theorem}

\noindent\textit{Proof sketch.}
The intuition for the erasure property is that no rule in the semantics uses
$S(v)$ for anything other than producing a probability in the $\pjudge{p}$
judgment,
and that those probabilities are only ever stored back into $S$.

The proof proceeds by inducting on the number of steps, $n$.
The base case ($n=0$) is trivial; for the inductive case, the goal is to show
that a single step preserves $H'$, $D'$, and $s'$ when the left-hand
probability map $S$ is replaced.
Two lemmas show that replacing $S$ with $S'$ in the expression judgments leads
to the same result value $V$ and, in the dynamically-tracked case, the same
tracking probability $p_d$.
Finally, structural induction on the small-step statement judgment shows that,
in every rule, the expression probability affects only $S$ itself.
