Approximate computing research is still in its early stages.
This dissertation re-examines traditional abstractions in hardware and
software and argues that they should include a notion of computational
quality.
It develops five principles for the design of approximation-aware
abstractions:

\paragraph{Application-Specific Result Quality}
A broad spectrum of applications come with correctness constraints that are
not binary:
there are better outputs and worse outputs.
But as with traditional correctness criteria, there is no single, universal
``soft'' quality criterion.
A key principle in this work is that programmers should express
\emph{quality metrics} to quantify an output's usefulness on a continuous
scale.
Quality metrics are essential not only to the design of tools that constrain
correctness,
but also to the empirical evaluation of any approximation technique.

\paragraph{Safety vs.~Quality}
The abstractions in this dissertation benefit from decomposing correctness
into two complementary concerns:
\emph{quality}, the degree of accuracy for approximate values, and
\emph{safety}, whether to allow any degree of approximation at all.
While this zero-versus-nonzero distinction may at first seem artificial, it
decomposes many intractable problems into two smaller problems that can be
tractably solved using different tools.
EnerJ (\chref{enerj}) and DECAF (\chref{decaf}) demonstrate this separation of
concerns:
information flow types are best suited for safety,
and constraint-solving numerical type inference is best suited for quality.
Using a single technique for both would be less effective.

\paragraph{Hardware--Software Co-Design}
Approximate computing is a cross-cutting concern.
While both hardware and software techniques hold promise, a good rule of thumb
is to \emph{never do hardware without software}.
\TODO{justify that}

\paragraph{Programming with Probabilistic Reasoning}

\paragraph{Granularity of Approximation}
