The research on approximate computing during this decade
has asked more questions than it has answered.
To bring approximation mainstream, the community will need to address a swath
of open problems.

\paragraph{Composition}
Most current tools for approximate programmability are stuck in a
whole-program paradigm.
ACCEPT's compiler analyses and auto-tuner machinery, from \chref{accept},
assume that they can observe the entire application at once.
Probabilistic assertions, from \chref{passert},
fundamentally describe whole-program properties: they constrain a chance that
an execution \emph{from program entry} has a certain property.
This whole-program perspective on result quality prevents approximate
computing from participating in some of the most powerful concepts in
programming: local abstractions, separation of concerns, and libraries.
A recent exception is Carbin \etal's Rely language~\cite{rely},
where accuracy is a relationship between module inputs and module outputs.
The next stage of research should continue to define what composition means in
an approximate context.

\paragraph{Connecting with everyday approximation}
Although the term is new, approximate computing is far from a new idea.
Approximation is a fundamental in some domains of computer science.
Digital signal processing pipelines incorporate accuracy parameters at every
stage;
work on real-time graphics gets good-enough results more cheaply than an ideal
renderer;
and there is an entire subfield in theoretical computer science that designs
approximation algorithms for intractable problems.
All of these approaches are approximations, but they look very different from
the kind of system-level approximations in this dissertation.
Programming models for approximate computing can learn lessons from these more
established disciplines.
And the new techniques developed for approximate computing may also be
portable in the opposite direction:
they could help bring programmability to areas where approximation has
traditionally been difficult to reason about.

\paragraph{Fault tolerance in high-performance computing}
While approximate computing is not the same as fault tolerance, there are
clear connections.
High-performance computing infrastructures are often large enough that silent
failures are a fact of life;
and, meanwhile, many HPC applications can tolerate some errors.
Approximate computing researchers should build a bridge to domain expertise in
HPC\@.
Ideally, approximate programming techniques could help
express the latent tolerance in
HPC systems while constraining the potential for numerical instability and
other failure modes.

\paragraph{Defining quality}
One of the principles of this research is that programs have
application-specific quality metrics.
Determining exactly what constitutes ``quality'' for a given application,
however, can be deceptively difficult.
Consider defects in images:
how many pixels can be wrong, and by what amount, before the user notices?
Are larger areas of slight discoloration better than smaller areas of more
intense errors?
What makes users care more about the quality of certain photographs than
others?
These questions are subjective, context sensitive, and poorly defined, but
they are critical to determining whether an approximation is successful.
For approximate computing to succeed, we need better methodologies for
deriving quality metrics.

\paragraph{Unifying with probabilistic programming}
Abstractions for approximate program usually need to incorporate probabilistic
semantics.
Recently, the programming languages research community has developed a focus
another area that combines programming with probability:
\emph{probabilistic programming languages} \TODO{cite something}.
So far, this direction has assumed a relatively narrow focus:
making it easier to express and work with machine-learning models.
But the two research areas should cross-pollinate:
techniques from one should apply to problems from the other.
Researchers should seek fundamental ideas that underly the two sets of
programmability challenges.

\vspace{\baselineskip}
\noindent
Even with these remaining challenges unsolved, approximate computing \TODO{has
an air of inevitability}.
