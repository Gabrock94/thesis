\section{Introduction}
\label{sec:intro}

Accuracy and reliability are fundamental tenets in computer system design.
Programmers can expect that the processor never exposes timing
errors, and networking stacks typically aim to provide reliable transports
even on unreliable physical media.
When errors do occasionally happen, we treat them as exceptional outliers, not
as part of the system abstraction.
Cosmic rays can silently flip bits in DRAM, for example,
but the machine will typically use error-correcting codes to
maintain the illusion for programmers that the memory is infinitely reliable.

But abstractions with perfect accuracy come at a cost.
Chips need to choose conservative clock rates to banish timing errors,
storage and communication channels incur error-correction overhead,
and parallelism requires expensive synchronization.

Meanwhile, many applications have intrinsic tolerance to inaccuracy.
Applications in domains like computer vision, media
processing, machine learning, and sensor data analysis already incorporate
imprecision into their design.
Large-scale data analytics focus on aggregate trends rather than the integrity
of individual data elements.
In domains such as computer vision and robotics, there are no perfect answers:
results can vary in their usefulness, and the output quality is always in
tension with the resources that the software needs to produce them.
All these applications are \emph{approximate
programs}: a range of possible values can be considered ``correct'' outputs
for a given input.

From the perspective of an approximate program, today's
systems are overprovisioned with accuracy. Since the program is resilient, it
does
not need every arithmetic operation to be precisely correct and every bit of
memory to be preserved at the same level of reliability.
\emph{Approximate computing} is a research agenda that seeks to better match
the accuracy in system abstractions with the needs of approximate
programs.


\paragraph{Disciplined approximation}

The central challenge in approximate computing is forging abstractions that
make imprecision \emph{controlled and predictable} without sacrificing its
efficiency benefits.
This goal of this dissertation is to design hardware and software around
approximation-aware abstractions that, together, make accuracy--efficiency
trade-offs attainable for programmers.
My work examines approximate abstractions
in the contexts of programming languages, computer architecture,
memory technologies, compilers, and software development
tools.

\section{Research Principles}

The work in this dissertation is organized around five principles
for the design of disciplined approximate abstractions.
These themes represent the collective findings of the concrete research
projects described later.
The principles are:
%
\begin{enumerate}
\item Result quality is an application-specific property.
\item Approximate abstractions should distinguish between safety properties
    and quality properties.
\item Hardware and software need to collaborate to reach realize the best
    potential of approximate computing.
\item Approximate programming models need to incorporate probability and
    statistics.
\item The granularity of approximation represents a trade-off between
    generality and potential efficiency.
\end{enumerate}
%
This section outlines each finding in more detail.

\subsection{Result Quality is Application Specific}

Since approximate computing navigates trade-offs between efficiency and result
quality, it needs definitions of both sides of the balance.
While \emph{efficiency} can have universal definitions---the time to
completion, for example, or the number of joules consumed---output
\emph{quality} is more subtle.
A key tenet in this work is is that applications must define ``output
quality'' case by case:
the platform cannot define quality without information from the programmer.

Following this philosophy, the system designs in this dissertation assume that
each approximate program comes with a \emph{quality metric}, expressed as
executable code, that scores the program's output on a continuous scale from
0.0 to 1.0.
A quality metric is the approximate-computing analog to a traditional software
\emph{specification}, which typically makes a binary decision about whether an
implementation is correct or incorrect.
Just as ordinary verification and optimization tools start from a
specification, approximate-computing tools start with a quality metric.

\subsection{Safety vs.~Quality}
\label{sec:princ:safety}

At first glance, a quality metric seems like sufficient information to specify
an application's constraints on approximation.
If the system can guarantee that a program's output will always have a quality
score above $q$, and the programmer decides that $q$ is good enough, what
could possibly go wrong?

In reality, it can be difficult or impossible for systems to prove arbitrary
quality bounds with perfect certainty.
Realistic tools can often only certify, for example, that any output's quality
score will be at least $q$ \emph{with high probability},
or that \emph{nearly every output} will exceed quality $q$ but rare edge cases
may do worse.
Even more fundamentally, it can be difficult for programmers to devise formal
quality metrics that capture every possible factor in their intuitive notion
of output quality.
Quality metrics can be simpler if their scope is narrowed to data where they
are most relevant: the pixels in an output image, for example, but not the
header data.

To that end, this dissertation embraces \emph{safety} as a separate concept
from quality.
A safety property, in the context of approximate computing, is a guarantee
that part of a program \emph{never} deviates from its precise counterpart---in
other words, that it matches the semantics of a traditional, non-approximate
system.
A quality property, in contrast, constrains the \emph{amount} that approximate
program components deviate from their precise counterparts.

In practice, we find a first-order distinction between \emph{no approximation at
all} and \emph{approximation of some nonzero degree} both simplifies reasoning
for programmers and makes tools more tractable.
My work has demonstrated that the two kinds of properties can be amenable to
very different techniques:
information flow tracking (\chref{enerj}) is appropriate for safety, for
example, but statistical hypothesis testing (\chref{passert}) is better for
quality.

\subsection{Hardware--Software Co-Design}

% Approximate computing techniques cover a broad range of approaches,
% from hardware to software.
Some of the most promising ideas unlock new sources of efficiency that are
only available in hardware:
exploiting the analog behavior of transistors, for example,
or mitigating the cost of error correction in memory modules.
Even hardware techniques, however, must be co-designed with their software
abstractions.
Because approximation techniques have subtle and wide-ranging effects on
program behavior,
models that apply them obliviously are unworkable.
Hardware--software co-design is crucial to any hardware approximation
technique.

\subsection{Programming with Probabilistic Reasoning}

Often, the most natural ways to reason about approximation and quality use
probabilistic tools.
Probabilistic reasoning lets us show show statements such as \emph{this output
will be high-quality with at least probability $P$} or \emph{an input randomly
selected from this distribution leads to a high-quality output with
probability $P'$}.
These probabilistic statements can simultaneously match the nondeterministic behavior
of approximate systems~\cite{truffle, npu, approxstorage}
and correspond to software quality criteria~\cite{decaf, passert}.

To support reasoning about quality, approximate
programming models need to incorporate abstractions for statistical behavior.
The DECAF type system, in \chref{decaf}, and
probabilistic assertions, in \chref{passert}, represent two
complementary approaches to reasoning about probabilistic quality
properties.

These approaches dovetail with the recent expansion of interest in
\emph{probabilistic programming languages}, which seek to augment
machine-learning techniques with language abstractions~\cite{church}.
Approximate programming systems can adapt lessons from this body of research.

\subsection{Granularity of Approximation}
\label{sec:princ:granularity}

The \emph{granularity} at which approximate computing applies is a
nonintuitive but essential factor in its success.
My and other researchers' work has explored approximation strategies at
granularities of both extremes:
fine-grained approximations that apply to individual instructions and
individual words of memory (e.g., Truffle~\cite{truffle});
and coarse-grained approximations that holistically transform entire
algorithms (e.g., neural acceleration~\cite{npu}).

An technique's granularity affects its
generality and its efficiency potential.
A fine-grained approximation can be very general:
an approximate multiplier unit, for example, can potentially apply to any
multiplication in a program.
But the efficiency gains are fundamentally limited to \emph{non-control} components,
since control errors can disrupt execution arbitrarily.
Even if an approximate multiplier unit can be very efficient,
the same technique can never
improve the efficiency of a branch, an address calculation,
or even the
scheduling of an approximate multiply instruction.
Approximations that work at a coarser granularity can address control costs,
so their potential gains are larger.
But these techniques tend to apply more narrowly:
techniques that pattern-match on algorithm structures~\cite{paraprox},
for example, place nuanced restrictions on the code they can transform.

The EnerJ language in \chref{enerj} was initially designed for
fine-grained hardware approximation techniques such as low-voltage functional
units.
While the granularity was good for programmability, it was bad for efficiency:
our detailed hardware design for fine-grained hardware
approximation~\cite{truffle} demonstrated limited benefit.
The ACCEPT compiler in \chref{accept} bridges the gap: its analysis
library and optimizations exploit the fine-grained annotations from EnerJ to
safely apply coarse-grained optimizations.


\section{Abstractions for Disciplined Approximation}

This dissertation supports the above research principles using a set of
concrete system designs.
The systems comprise programming-language constructs that express
applications' resilience to approximation along with system-level
techniques for exploiting that latent resilience to gain efficiency.
This section serves as an overview of the interlocking designs;
Parts~\ref{part:programming} and~\ref{part:systems} give the full details.

\subsection{Controlling Safety and Quality}

The first set of projects consists of language abstractions that give
programmers control over safety and quality in approximate programs.

\subsubsection{Information Flow Tracking for General Safety}

EnerJ, described in \chref{enerj}, is a type system for enforcing safety in
the presence of approximation.
The key insight in EnerJ is that approximate programs tend to consist of two
intermixed kinds of storage and computation:
critical control components
and non-critical data components.
The latter, which typically form the majority of the program's execution,
are good candidates for approximation,
while the former should be protected from error and carry traditional
semantics.

EnerJ lets programmers enforce a separation between critical and non-critical
components.
It uses a type system that
borrows from static data
flow systems for security~\cite{jif, infflow-survey} to provide a static
noninterference guarantee for precise data.
EnerJ extends Java with two type qualifiers, \code{@Approx} and
\code{@Precise}, and uses a subtyping relationship to prevent
approximate-to-precise information flow.
Using EnerJ, programmers can rely
on a proof that data marked as precise remains untainted by the errors arising
from approximation.

A key design goal in EnerJ is its \emph{generality:}
the language aims to encapsulate a range of approximation strategies under a
single abstraction.
Its type system covers approximate storage via the types of variables and
fields;
approximate processor logic via overloading of arithmetic operators;
and even user-defined approximate algorithms using dynamic method dispatch
based on its approximating qualifiers.

EnerJ addresses safety, not quality:
a variable with the type \code{@Approx float} can be arbitrarily incorrect and
EnerJ does not seek to bound its incorrectness.
By leaving the complementary concern of controlling quality to separate
mechanisms, EnerJ keeps its type system simple.

\subsubsection{Extending EnerJ with Probability Types}

DECAF, in \chref{decaf}, extends EnerJ's type-based approach to safety with
quality guarantees.
The idea is to generalize the original \code{@Approx} type qualifier to a
parameterized qualifier \code{@Approx($p$)}, where $p$ dictates the
\emph{degree} of approximation.
Specifically, in DECAF, $p$ is the lower bound on the probability that a value
is \emph{correct:} that the value in an approximate execution equals its
counterpart in a completely precise execution of the same program.
DECAF defines sound type rules for introducing and propagating these
correctness probabilities.

DECAF's added sophistication over EnerJ's simple two-level system comes at a
cost in complexity:
a type system that requires probability annotations on every expression,
however, would quickly become infeasible for programmers.
To mitigate annotation overhead, DECAF adds type inference.
Sparse probability annotations on the inputs and outputs of coarse-grained
subcomputations are typically enough for DECAF's inference system to determine
the less-intuitive probabilities for intermediate values.
Crucially, DECAF places no constraints on where programmers can write explicit
annotations:
developers can write probabilities where they make the most sense and leave
the remaining details to the compiler.

DECAF addresses the limitations of a conservative quality analysis using
an optional dynamic-tracking mechanism.
The inference system also allows efficient code reuse by specializing
functions according to the accuracy constraints of their calling contexts.

\subsubsection{Probabilistic Assertions}

DECAF's approach to controlling quality achieves strong probabilistic
guarantees by constraining the range of possible approximation strategies:
it works only with techniques where errors appear at an operation granularity;
whey they occur randomly but rarely; and when the error probability is
independent of the input values.

A complementary project takes the opposite approach:
it accommodates any probability distribution, but it offers weaker
guarantees.
The idea is to use statistical hypothesis tests to prove properties up to a
\emph{confidence level:}
to allow a small probability of ``verifying'' a false property.

The technique is based on a new language construct called a
\emph{probabilistic assertion}.
The construct is analogous to a traditional assertion: \code{assert $e$}
expresses that the expression $e$ must always be true.
A probabilistic assertion:
%
\begin{lstlisting}
passert $e$, $p$, $c$
\end{lstlisting}
%
indicates that $e$ must
be true with at least probability $p$, and the system has to prove the
property at confidence level $c$.
These assertions can encode important quality properties in approximate
programs, such as bounds on the frequency of ``bad'' pixels produced by an
image renderer.
The same construct is useful in other domains where probabilistic behavior is
essential, such as when dealing with noisy sensors.

\chref{passert} describes probabilistic assertions in more detail
along with a workflow for verifying them efficiently.
The verifier uses a symbolic-execution technique to extract a representation
of a program's probabilistic behavior: a Bayesian network.
The verifier can optimize this Bayesian-network representation using
off-the-shelf statistical properties that are difficult to apply to the
original program code.
The complete workflow can make probabilistic-assertion verification dozens of
times faster to check than a naive stress-testing approach.

\subsection{Exploiting Resilience for Efficiency}

The second category of research is on the implementation of systems that
exploit programs' tolerance for approximation to improve efficiency.
This dissertation describes two projects: an architectural technique and an
end-to-end compiler toolchain.
A primary concern in both systems is exposing an abstraction that fits with
the safety and quality constraints introduced in the above language
abstractions.

\subsubsection{Approximate Storage for Solid-State Memory Technologies}

One system design, detailed in \chref{approxstorage}, exploits builds on trend in hardware technologies.
It exploits unique properties of new solid-state memories,
such as flash and phase-change memory,
to implement two orthogonal trade-offs between resource and accuracy.

The first technique recognizes that the underlying material in these memory
technologies is analog.
Traditional designs build a clean digital abstraction on top of a
fundamentally analog memory cell.
Our technique addresses the cost of that digital abstraction by letting
applications opt into stochastic data retention.

The second technique embraces resistive memory technologies' tendency to wear
out.
Ordinarily, architectures need to detect failed memory blocks and avoid
storing data in them---limiting the memory module's useful lifetime.
Instead, in the context of an approximate application, we can harvest the otherwise-unusable
blocks and store approximate data in them.

Both strategies need a new set of common CPU and operating-system interfaces
to let software communicate error resilience and bit layout information.
We develop these abstractions to match the structure and semantics of EnerJ.

\subsubsection{ACCEPT: An Approximate Compiler}

The final system design takes a different tactic:
rather than simulating hypothetical hardware,
the idea is to build a practical infrastructure for experimenting with
approximation in the nearer term.
\chref{accept} introduces ACCEPT, an open-source compiler workflow designed both for
practitioners, to try out approximation techniques on their code,
and for researchers, to prototype and evaluate new ideas for approximation
strategies.

The first challenge that ACCEPT faces is to bridge the granularity gap (see
\secref{princ:granularity}, above).
EnerJ's fine-grained annotations can be more general and easier to apply to
programs,
but coarse-grained optimizations can offer better efficiency
gains---especially in the pure-software domain.
ACCEPT's interactive optimization architecture,
compiler analysis library,
and auto-tuner infrastructure help connect fine-grained safety annotations to
coarse-grained optimizations.

ACCEPT also addresses a second persistent challenge in approximate
programmability:
balancing automation with programmer control.
Fully manual approximation can be tedious and error prone,
but fully automatic systems can also frustrate developers by isolating them
from decisions that can break their code.
ACCEPT relies on the distinction between quality and safety (see
\secref{princ:safety}) reconcile the extremes.
Type annotations resembling EnerJ's enforce safety, but programmers are kept
in the loop with an interactive optimization workflow to rule out unexpected
quality effects.
Together, the systems leverage the best of both factors:
programmer insight for preserving application-specific properties,
and automatic compiler reasoning for identifying obscure data flows.

\section{Other Work}

The work in this document is intimately connected to other research I
collaborated on while at the University of Washington.
While this dissertation does not fully describe these related projects, their
influence is evident in the trajectory of projects that do appear here.
For context, this section describes a handful of other projects on approximate
hardware and developer tools.


\subsection{An Approximate CPU and ISA}

Truffle is a processor architecture that implements EnerJ's semantics to save
energy~\cite{truffle}. It uses a secondary, subcritical voltage that allows timing errors in
a portion of the logic and retention errors in a portion of the SRAM.

To expose the two voltages to software, we
designed an ISA extension that includes a notation of abstract approximation.
The code can choose dynamically to enable approximation
per instruction, per register, and per cache line.
A key challenge in the design was supporting an ISA that could efficiently
support an EnerJ-like programming model, where the precise and approximate
components of a program remain distinct but interleave at a fine grain.

Our simulation of the Truffle design space yielded results ranging a 5\%
energy consumption \emph{increase} to a 43\% reduction.
These results emphasize the efficiency limits of very fine-grained
approximation (see the granularity principle in \secref{princ:granularity}).
Even in a maximally approximate program---in which every
arithmetic instruction and every byte of memory is marked as
approximate---much of Truffle's energy is spent on precise work. Fetching
code, scheduling instructions, indexing into SRAMs, computing addresses, and
tracking precision state all must be performed reliably.
Modern processors spend as much energy on
control as they do on computation itself, so any technique that optimizes only
computation will quickly encounter Amdahl's law.

The Truffle work in appears in the thesis of Hadi
Esmaeilzadeh~\cite{hadi-thesis}.


\subsection{Neural Acceleration}
\label{sec:npu}

\emph{Neural acceleration} is a technique that explores the opposite end of
the granularity spectrum~\cite{npu, snnap}.
The idea is to use machine learning to imitate a portion of a computation by
observing its input--output behavior.
Then, we build a configurable hardware accelerator to efficiently execute the
learned model in place of the original code.
Our specific design uses neural networks:
since neural networks have
efficient hardware implementations, the transformed function can be much
faster and lower-power than the original code.

The coarse granularity pays off in efficiency: our simulations demonstrated a
3$\times$ average energy reduction.
But the coarser granularity comes at a cost of programmer visibility and
control.
Since the NPU technique treats the target code as a black box, the
programmer has no direct influence over the performance and accuracy of the
resulting neural network.
These conflicting objectives demonstrate the need for techniques that bridge
the granularity gap.

The original neural acceleration work also appears in Hadi Esmaeilzadeh's
thesis~\cite{hadi-thesis}.
I also worked on a recent extension of the idea for programmable
logic~\cite{snnap}.


\subsection{Monitoring and Debugging Quality}

Many approaches to making approximation programmable focus on proving
conservative, static bounds.
As in traditional software development,
approximate computing also needs complementary dynamic techniques.
To this end, I contributed to a pair of techniques for dynamically controlling
result quality~\cite{approxdebug}.

The first dynamic system is a framework for monitoring quality in deployment.
The goal is to raise an exception whenever the program produces a ``bad''
output.
While the ideal monitoring system would directly measure the quality
degradation of every output,
perfect measurement is infeasible expensive for run-time deployment.
Our framework provides a range of techniques for specific scenarios where we
can make monitoring cheap enough to be feasible.

The second system is a debugging tool.
The idea is that certain subcomputations can be more critical to important to
quality than others, but that this difference is not necessarily obvious to
programmers.
The tool identifies and blames specific approximation decisions in a large
codebase when they are responsible for too much quality degradation.

The work on dynamic quality analysis appears in the thesis of
Michael F.\ Ringenburg~\cite{ringenburg-thesis}.


\section{Organization}

The next chapter is a literature survey of work on efficiency--accuracy
trade-offs.
Historical context is particularly important to this dissertation because the
fundamental idea of exchanging accuracy for returns in efficiency is so old:
analog computers and floating-point numbers, for example, are prototypical
examples of
approximate-computing strategies.

Parts~\ref{part:programming} and~\ref{part:systems} form the core of the
dissertation.
They comprise five independent but interlocking research projects that
together build up abstractions for making approximate computing both tractable
and efficient.
%
Part~\ref{part:programming} describes three approaches to abstracting
approximation in programming languages:
EnerJ, a type system that uses type qualifiers to make approximation safe;
DECAF, an extension of EnerJ that adds probabilistic reasoning
about the likelihood that data is correct;
and probabilistic assertions, a strategy for efficiently verifying complex
probabilistic properties via sampling.
%
Part~\ref{part:systems} describes two system designs for implementing
efficiency--accuracy trade-offs:
a hardware architecture that exploits the nuances of resistive memory
technologies such as phase-change memory;
and an open-source compiler toolkit that provides the scaffolding to quickly
implement new approximation strategies while balancing programmability with
approximation's potential benefits.

Finally, \plurchref{conclusion} and~\ref{ch:future} look forward and
backward, respectively.
The retrospective chapter distills lessons from the work in this dissertation
about approximate computing and hardware--software co-design in general, and
the prospective chapter suggests next steps for bringing approximation into
the mainstream.

This dissertation also includes appendices that formalize the
programming-languages techniques in Part~\ref{part:programming} and prove
their associated theorems.


\section{Previously Published Material}

This dissertation comprises work published elsewhere in conference papers:

\begin{itemize}
\item \chref{enerj}:
\textit{EnerJ: Approximate Data Types for Safe and General Low-Power
Computation.}
Adrian Sampson, Werner Dietl, Emily Fortuna, Danushen Gnanapragasam, Luis Ceze, and Dan Grossman.
In Programming Language Design and Implementation (PLDI), 2011.
\cite{enerj}

\item \chref{decaf}:
\textit{Probability Type Inference for Flexible Approximate Programming.}
Brett Boston, Adrian Sampson, Dan Grossman, and Luis Ceze.
To appear in OOPSLA 2015.
\cite{decaf}

\item \chref{passert}:
\textit{Expressing and Verifying Probabilistic Assertions.}
Adrian Sampson, Pavel Panchekha, Todd Mytkowicz, Kathryn McKinley, Dan Grossman, and Luis Ceze.
In Programming Language Design and Implementation (PLDI), 2014.
\cite{passert}

\item \chref{approxstorage}:
\textit{Approximate Storage in Solid-State Memories.}
Adrian Sampson, Jacob Nelson, Karin Strauss, and Luis Ceze.
In the IEEE/ACM International Symposium on Microarchitecture (MICRO), 2013.
\cite{approxstorage}
\end{itemize}
