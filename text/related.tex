Approximate computing research combines insights from hardware engineering,
architecture, system design, programming languages, and even application
domains like machine learning. I have confined this section to describe work
that considers the possibility of exposing errors, incorrectness, and
uncertainty to applications. I will not discuss the large body of work on
traditional fault tolerance, which attempts to recover from faults to compute
results without error.

\section{Application-Level Error Tolerance}
\label{sec:related:studies}

Many authors have identified the property of error tolerance in existing
``soft''
applications. A large class of studies have examined this property by
injecting errors into certain parts of applications and assessing the
execution quality in terms of both crashes and output fidelity~\cite{li06,
li07, li08, dekruijf-selse09, wong-selse06, palem-arcs, freton, besteffort,
yeh, thaker-iiswc06, efc, llfi, chippa-dac}.

It is a consensus among many of these studies that different parts of the
application have different impacts on reliability and fidelity. Some conclude
that there is a useful distinction between critical and non-critical program
points (typically instructions)~\cite{palem-arcs, thaker-iiswc06, flikker,
llfi}.
The work tends to assume that at least the instructions involved in control
flow are critical~\cite{thaker-iiswc06}. In my work, I use this idea of
distinguishing between critical and noncritical components to help the
programmer express approximations with bounded effects on the program as a
whole. In each of the approximate system components I have designed, the
execution substrate supports precise (traditional) and relaxed operation to
exploit this distinction. An important question, however, is the granularity
at which to switch between these two modes. \secref{princ:granularity} discusses
this challenge in more detail.

This class of work also typically attempts to classify the kinds of programs
that can tolerate error injection gracefully. For example, papers have focused
on video~\cite{freton}, recognition and mining~\cite{besteffort}, physical
simulation~\cite{yeh}, and optimization~\cite{hogwild}.

Aside from existing software, some studies have evaluated error-resilience in
integrated circuit designs~\cite{breuer, scalable-effort-hardware}. These
studies focus on the kinds of applications that are typically implemented with
application-specific circuits: media codecs, numerical kernels, and digital
signal processing.


\section{Exploiting Resilience in Architecture}

The existence of a large class of error-resilient applications has motivated
research into techniques that exploit this property to reduce resource usage.
Specifically, designs have emerged that extend circuits and processor
architectures to trade off error for energy, time, manufacturing yield, or
verification complexity.

Since floating-point operations are both expensive and inexact, they make a
profitable target for this research direction. Researchers have designed units
that dynamically adapt mantissa width~\cite{bitwidthred, hierarchfpu}, ``fuzzily'' memoize
similar arithmetic computations~\cite{fuzzymemo}, or tolerate timing
errors~\cite{palem-adders, impact, kumarhpca, hizli, adder-metrics}.
Conversely, fault-tolerant functional units can help avoid catastrophic
miscalculations while allowing occasional small errors~\cite{uva-adder}.
Alternative number
representations work in tandem with relaxed functional units to bound the
numerical error that can result from bit flips~\cite{stanleymarbell}. SRAM
structures spend significant static power on retaining data, so they represent
another opportunity for fidelity trade-offs~\cite{hybrid-sram}. Similarly,
DRAM structures can reduce the power spent on refresh cycles where bit flips
are allowed~\cite{flikker, sparkk}.
In persistent memories where storage cells can wear out, approximate systems
can reduce the number of bits they flip to lengthen the useful device
lifetime~\cite{fang-pcm}.
Similarly, low-power writes to memories like flash can exploit its
probabilistic properties while hiding them from software~\cite{halfwits,
powerfade, flash-retention-relax}.

Some recent work has also proposed general techniques for making quality trade-offs
when synthesizing and optimizing general hardware
circuits~\cite{lossysynthesis, palem-pruning, rahimi, axilog, miao-thesis}.

As a dual to adding errors in some circuits, some researchers have
explored differential fault protection in the face of universally unreliable
circuits. As process sizes continue to shrink, it is likely that reliable
transistors will become the minority; redundancy and checking will be
necessary to provide reliable operation~\cite{li-asplos08}. Circuit design
techniques have been proposed that reduce the cost of redundancy by providing
it selectively for certain instructions in a CPU~\cite{wreft} or certain
blocks in a DSP~\cite{unequal-protection, ant}.
Researchers at Wisconsin have applied relaxed error tolerance at the
architecture level to GPUs~\cite{palframan-gpu}.
Other work has used criticality information to selectively allocate
software-level error detection and correction
resources~\cite{khudia-tolerance, shi-cal}.

Aside from designing fundamentally approximate circuits, a different direction
introduces approximation by relaxing traditional microarchitectural
mechanisms.
Notably, ``soft coherence'' relaxes intercore
communication~\cite{softcoherence},
and load value approximation~\cite{lva-sanmiguel, lva-thwaites} approximates
numerical values instead of fetching them from main memory on cache misses.

Recent work has proposed system organizations that apply approximation at a
coarser grain.
Yetim et al., for instance,
describe a technique that uses coarse-grained quality techniques to
allow errors even in processor control logic~\cite{martonosi-date, commguard}.
Duwe~\cite{duwe-thesis} proposes run-time coalescing of approximate and
precise computations to reduce the overhead of switching between modes.
Other work allocates approximation among the lanes of a SIMD
unit~\cite{tabsh}.

Near-threshold voltage domains also present a new opportunity for embracing
unpredictable circuit operation~\cite{soft-ntc}.




\section{Exploiting Resilience with Program Transformations}
\label{sec:related:software}

Aside from system-level accuracy trade-offs, there are opportunities for
adapting \emph{algorithms} to execute with varying precision. Algorithmic
quality--complexity trade-offs are not new: approximation algorithms are a
well-studied area of complexity theory and many domains, notably real-time
vision and graphics, have ad-hoc ``cheap'' implementations of computations
that are prohibitively expensive to compute exactly.

In contrast to these
traditional views on approximation, recent work has sought to provide
tools for transforming real programs---as opposed to high-level
algorithmic specifications---to explore practical relaxations.
Transformations include removing portions of a program's dynamic execution
(termed \emph{code perforation})~\cite{perforation-fse}, unsound
parallelization of serial programs~\cite{quickstep}, eliminating
synchronization in parallel programs~\cite{dubstep, races-ibm, hogwild},
identifying and adjusting parameters that control output
quality~\cite{dynamicknobs}, randomizing portions of deterministic
programs~\cite{zhu-popl12, sasa-sas11}, dynamically choosing between
different programmer-provided implementations of the same
specification~\cite{green, virus, petabricks, taco-soc, ansel-autotuning}, and replacing pure computations with invocations
of a trained neural network~\cite{benchnn, temam-isca, emeuro}.

Central to each of these program transformation techniques are the questions
of automation and quality. For a relaxation to be generally useful, it should
be applicable automatically with only minimal programmer involvement but
should still result in transformed code that the programmer is happy with.
Recently, a set of techniques has been proposed to help constrain the space
of possible transformations to those that result in a profitable
quality trade-off. Quality of service profiling~\cite{qosprof} uses many
executions to identify parts of a program that are likely to be good
candidates for transformation. Some verification tools and proof systems help
the programmer prove relationships between the original program and a
candidate relaxed version~\cite{carbin-pldi, carbin-races, carbin-pepm,
rice-transformation-semantics}.
Another approach constrains the programming model to help express programs
that refine their results as they run longer, permitting quality trade-offs in
hard real-time settings~\cite{chung90}.

Recent work from Michigan combines pattern-matching code transformations with
dynamic monitoring to provide approximation trade-offs for GPU
workloads~\cite{paraprox, sage}.
Similarly, Sartori and Kumar~\cite{herding} propose to transform GPU programs
to reduce their control divergence at the cost of correctness.

Recently, a research direction has developed in \emph{automated program
repair} and other approaches to heuristically patching software according to
programmer-specified criteria.
These techniques are typically approximate in that they abandon a traditional
compiler's goal of perfectly preserving the original program's semantics.
Notably, Schulte et~al.~\cite{schulte} propose to use program evolution to
optimize for energy.

Precimonious~\cite{precimonious} addresses the problem of choosing appropriate
floating-point widths, which amount to a trade-off between numerical accuracy
and space or operation cost.
Similarly, STOKE's floating-point extension~\cite{stoke-fp} synthesizes new
versions of floating-point functions from scratch to meet different accuracy
requirements with optimal efficiency.

Neural acceleration is a recent technique treading code as a black box and
transforming it to a neural-network representation.
It is, at its core, an algorithmic transformation, but it integrates tightly
with hardware support: a digital accelerator, analog circuits, FPGAs, or
GPUs~\cite{npu, anpu, snnap, neuralgpu}.
See \secref{npu} for a more detailed overview of neural acceleration.


\section{Dynamically Controlling Approximation}

Program-level transformations can also help detect egregious errors while
allowing small, inconsequential disruptions~\cite{lwc, approxdebug}.
A variety of techniques propose mechanisms for run-time or profiling feedback to adapt
approximation parameters~\cite{dynamicknobs, green, approxit, ansel-autotuning}.
\TODO{describe these in a little more detail}

\section{Languages for Expressing Approximation}

Recently, language constructs that express and constrain
approximation have become a focus in the programming-languages research
community.
Relax~\cite{relax} is a language with ISA support for tolerating architectural
faults in software.
Rely~\cite{rely} uses specifications that relate the reliability of the input
to an approximate region of code to its outputs.

A related set of recent approximate-programming tools attempt to \emph{adapt}
a program to meet accuracy demands while using as few resources as possible.
Chisel~\cite{chisel} is an extension to Rely that searches for the subset of
operations in a program that can safely be made approximate.
ExpAX~\cite{expax-tr} finds safe-to-approximate operations automatically and
uses a metaheuristic to find which subset of them to actually approximate.

\TODO{cite \cite{energytypes}}


\section{Exploiting Resilience in Other Systems}

While architecture optimizations and program transformations dominate the
field of proposed exploitations of approximate software, some recent work has
explored the same trade-off in other components of computer systems. Network
communication, with its reliance on imperfect underlying channels, exhibits
opportunities for fidelity trade-offs~\cite{softcast, luo-globecom, apex,
smpmup2006}. Notably, SoftCast~\cite{softcast} transmits images and video by
making the signal magnitude directly proportional to pixel luminance. BlinkDB,
a recent instance research on \emph{approximate query answering},
is a database system that can respond to queries that include a required
accuracy band on their output~\cite{blinkdb}.
Uncertain{\textless}T{\textgreater}~\cite{uncertaint} and Lax~\cite{lax}
propose to expose the approximate, probabilistic behavior of sensors at the
language and API levels.
By eschewing redundancy and
recovery in a fault-tolerant distributed system, researchers at Wisconsin were
able to make quality trade-offs in a supercomputing
setting~\cite{dekruijf-icpp}.


\section{Probabilistic Computation}

Several research projects have proposed radically different models of
computation that, rather than adapting existing software to expose its error
resilience, are based exclusively on probabilistic reasoning. Most
prominently, Probabilistic CMOS~\cite{pcmos, pcmos-cacm, palem-dac-position}
and stochastic processors~\cite{stochasticproc} expose the probabilistic
behavior of transistors as part of their ISA.
Other preliminary efforts from MIT~\cite{batesmit, lyric, mansinghka-circuits} and
Stanford~\cite{ersa} fall into the same category.

\section{Probabilistic Semantics}

Researchers have proposed several languages and tools to
help developers better reason about and describe real-world
probabilistic data, computation, and models~\cite{BBGR13,
  wingate-lightweight, church, chaganty, pfeffersample, pmonad,
  infernet, probdsl,uncertaint}.
The probability monad captures a variable's discrete probability
distribution in functional programs~\cite{pmonad}.
\TODO{this one specific call-out is awkward}

Sankaranarayanan et al.~\cite{sriram-pldi} check assertions in
programs that produce probabilistic models using symbolic execution
and polyhedral volume estimation. The \texttt{estimateProbability} construct
queries the probability of an outcome, which resembles a probabilistic
assertion's
specification of the correct outcome.

Statistical model checking bounds verification error on 
problems where state-space explosion makes exact numerical
verification intractable (see Legay and Delahaye's
survey~\cite{legay10}).  Model checking~\cite{Clarke} provides formal guarantees,
usually expressed in temporal logic for finite state
based models, often of hardware. For example, the
PRISM tool performs statistical verification of real-time
systems~\cite{KNP11}. Our work borrows the idea of hypothesis
testing to bound error in verification~\cite{Younes,Younes20061368}
and
relies on efficient sampling to avoid
the need for exhaustive state space exploration.

Similarly, recent work has sought to express
uncertainties in the context of traditional, imperative programming
languages~\cite{uncertaint}. These programming models incorporate algorithmic
probabilities---such as those inherent in sensor readings or machine learning
model parameters---but do not yet address the computational probabilities of
techniques like PCMOS and stochastic processors.

Kozen~\cite{kozen} recognizes the need for semantics for programs
that use randomness during execution.
That work provides two semantics for a simple probabilistic
language---one that models sampling and one that computes on probability
distributions directly---and proves them equivalent.
Similarly, we prove equivalence between sampling an original program and
sampling its extracted Bayesian network representation.
Kozen predates the coinage and popularization of Bayesian networks, so
the semantics in that work are very different from the graphical-model approach
presented here.
Our Bayesian-network representation enables statistical optimizations that
make probabilistic assertion verification efficient.
\TODO{this paragraph has dangling references to the passert paper}

\subsection{Probabilistic Programming Languages}

The field of \emph{probabilistic programming} seeks to enable efficient
construction and querying of statistical models~\cite{BBGR13, wingate-lightweight,
  church, chaganty, pfeffersample, probdsl, koller}.  Experts write
generative models as programs and then inference algorithms answer questions
about the model's parameters. The canonical probabilistic programming example
answers, ``given that the grass is wet, was it due to rain or the sprinkler?''




\section{Robustness Analysis}

As the studies in Section~\ref{sec:related:studies} repeatedly find, error
tolerance varies greatly in existing software, both within and between
programs. Several program analysis approaches have been proposed to evaluate
and enhance error resilience properties. SJava analyzes programs to prove that
errors only temporarily disrupt the execution path of a program~\cite{sjava}.
Program smoothing~\cite{smoothing-cav, smoothing-pldi, smoothing-fse} and
``robustification''~\cite{robustification} both find continuous, mathematical
functions that resemble the input--output behavior of numerical programs.
Auto-tuning approaches can help empirically identify error-resilient
components~\cite{asac}.
Finally, Cong and Gururaj describe a technique for automatically
distinguishing between critical and non-critical instructions for the purpose
of selective fault tolerance~\cite{cong-iccad}.
