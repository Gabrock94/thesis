\emph{Approximate computing} is the idea that we are hindering computer
systems' efficiency by demanding too much accuracy from them.
While predictability and reliability are crucial for some tasks, many
modern applications are fundamentally approximate.
Perfect answers are unnecessary or even impossible in domains such as
computer vision, machine learning,
speech recognition, search, graphics, and physical simulation.
Today's systems waste time, energy, and complexity to provide uniformly
pristine operation for these applications that do not require it.

Resilient applications are not, however, a license for computers to abandon
predictability in favor of arbitrary errors.
We need abstractions that incorporate approximate operation in a
\emph{disciplined} way.
Application programmers should be able to exploit these richer abstractions to
treat accuracy as a resource and trade it off for more ``traditional''
resources like time, space, or energy.

This dissertation explores new abstractions for approximate computing across
hardware and software.
It develops these abstractions from two perspectives:
from the point of view of \emph{programmers}, where the challenge is
constraining imprecision to make it acceptable,
and from a \emph{system} perspective, where the goal is to exploit programs'
constraints to improve efficiency.
For approximate programming, this dissertation proposes:
%
\begin{itemize}
\item
a type system based on static information flow tracking that separates an
application's error-resilient components from its critical control structures;
\item
an extended type system that restricts the probability that a value
is incorrect, along with type inference and optional dynamic tracking for
these probabilities; and
\item
a construct for expressing probabilistic constraints on programs along with a
technique for verifying them efficiently using symbolic execution and
statistical properties.
\end{itemize}
%
For approximate execution, it describes:
%
\begin{itemize}
\item
two mechanisms for trading off accuracy for density, performance, energy, and
lifetime in solid-state memory technologies, and
\item
an end-to-end compiler framework for exploiting approximation on
commodity hardware, which also serves as research infrastructure for
experimenting with new approximation ideas.
\end{itemize}
